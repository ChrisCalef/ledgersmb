<?lsmb FILTER latex -?>
\documentclass{scrartcl}
\usepackage[utf8]{inputenc}
\usepackage{tabularx}
\usepackage[letterpaper,top=2cm,bottom=1.5cm,left=1.1cm,right=1.5cm]{geometry}
\usepackage{graphicx}

\begin{document}

\pagestyle{myheadings}
\thispagestyle{empty}

\fontfamily{cmss}\fontsize{10pt}{12pt}\selectfont

\newsavebox{\ftr}
\sbox{\ftr}{
  \parbox{\textwidth}{
  \tiny
   \rule[1.5em]{\textwidth}{0.5pt}
Items returned are subject to a 10\% restocking charge.
A return authorization must be obtained from <?lsmb company ?> before goods are
returned. Returns must be shipped prepaid and properly insured.
<?lsmb company ?> will not be responsible for damages during transit.
  }
}

<?lsmb INCLUDE letterhead.tex ?>

% Breaking old pagebreak directive
%<?xlsmb pagebreak 65 27 37 ?>
%\end{tabularx}
%
%\newpage
%
%\markboth{<?xlsmb company ?>\hfill <?xlsmb ordnumber ?>}{<?xlsmb company ?>\hfill <?xlsmb ordnumber ?>}
%
%\begin{tabularx}{\textwidth}{@{}rlXllrrl@{}}
%  \textbf{Item} & \textbf{Number} & \textbf{Description} & \textbf{Serial Number} & & \textbf{Qty} & \textbf{Ship} & \\
%<?xlsmb end pagebreak ?>


\vspace*{0.5cm}

\parbox[t]{.5\textwidth}{
\textbf{Ship To}} \hfill

\vspace{0.3cm}

\parbox[t]{.5\textwidth}{
  
<?lsmb shiptoname ?>

<?lsmb shiptoaddress1 ?>

<?lsmb shiptoaddress2 ?>

<?lsmb shiptocity ?>
<?lsmb IF shiptostate ?>
\hspace{-0.1cm}, <?lsmb shiptostate ?>
<?lsmb END ?>
<?lsmb shiptozipcode ?>

<?lsmb shiptocountry ?>
}
\parbox[t]{.5\textwidth}{
  <?lsmb shiptocontact ?>
  
  <?lsmb IF shiptophone ?>
  Tel: <?lsmb shiptophone ?>
  <?lsmb END ?>
  
  <?lsmb IF shiptofax ?>
  Fax: <?lsmb shiptofax ?>
  <?lsmb END ?>
  
  <?lsmb shiptoemail ?>
}
\hfill

\vspace{1cm}

\textbf{P A C K I N G} \parbox{0.3cm}{\hfill} \textbf{L I S T}
\hfill

\vspace{1cm}

\begin{tabularx}{\textwidth}{*{7}{|X}|} \hline
  \textbf{Invoice \#} & \textbf{Order \#} & \textbf{Date} & \textbf{Contact}
  <?lsmb IF warehouse ?>
  & \textbf{Warehouse}
  <?lsmb END ?>
  & \textbf{Shipping Point} & \textbf{Ship via} \\ [0.5em]
  \hline
  
  <?lsmb invnumber ?> & <?lsmb ordnumber ?>
  <?lsmb IF shippingdate ?>
  & <?lsmb shippingdate ?>
  <?lsmb ELSE ?>
  & <?lsmb transdate ?>
  <?lsmb END shippingdate ?>
  & <?lsmb employee ?>
  <?lsmb IF warehouse ?>
  & <?lsmb warehouse ?>
  <?lsmb END ?>
  & <?lsmb shippingpoint ?> & <?lsmb shipvia ?> \\
  \hline
\end{tabularx}
  
\vspace{1cm}
  
\begin{tabularx}{\textwidth}{@{}rlXllrrl@{}}
  \textbf{Item} & \textbf{Number} & \textbf{Description} & \textbf{Serial Number} & & \textbf{Qty} & \textbf{Ship} & \\

<?lsmb FOREACH number ?>
<?lsmb lc = loop.count - 1 ?>
  <?lsmb runningnumber.${lc} ?> &
  <?lsmb number.${lc} ?> &
  <?lsmb description.${lc} ?> &
  <?lsmb serialnumber.${lc} ?> &
  <?lsmb deliverydate.${lc} ?> &
  <?lsmb qty.${lc} ?> &
  <?lsmb ship.${lc} ?> &
  <?lsmb unit.${lc} ?> \\
<?lsmb END ?>
\end{tabularx}


\parbox{\textwidth}{
\rule{\textwidth}{2pt}

\vspace{12pt}

<?lsmb notes ?>

}

\vfill

\rule{\textwidth}{0.5pt}

\usebox{\ftr}

\end{document}
<?lsmb END ?>
