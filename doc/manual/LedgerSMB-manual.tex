%% LyX 1.4.2 created this file.  For more info, see http://www.lyx.org/.
%% Do not edit unless you really know what you are doing.
\documentclass[english]{article}
\usepackage[T1]{fontenc}
\usepackage[latin1]{inputenc}
\IfFileExists{url.sty}{\usepackage{url}}
                      {\newcommand{\url}{\texttt}}

\newcommand{\version}{1.3}
\makeatletter

%%%%%%%%%%%%%%%%%%%%%%%%%%%%%% LyX specific LaTeX commands.
\providecommand{\LyX}{L\kern-.1667em\lower.25em\hbox{Y}\kern-.125emX\@}
%% Bold symbol macro for standard LaTeX users
\providecommand{\boldsymbol}[1]{\mbox{\boldmath $#1$}}

%% Because html converters don't know tabularnewline
\providecommand{\tabularnewline}{\\}

%%%%%%%%%%%%%%%%%%%%%%%%%%%%%% User specified LaTeX commands.

\usepackage{metatron}

\renewcommand{\abstractname}{Executive Summary}
\title{LedgerSMB Manual v. \version}
\author{The LedgerSMB Core Team}
\date{\today}


\usepackage{babel}
\makeatother
\begin{document}
\maketitle

Except for the included scripts (which are licensed under the GPL v2 or later),
the following permissive license governs this work.

Copyright 2005-2012 The LedgerSMB Project. 

Redistribution and use in source (\LaTeX) and 'compiled' forms (SGML,
HTML, PDF, PostScript, RTF and so forth) with or without modification, are
permitted provided that the following conditions are met:

\begin{enumerate}
\item Redistributions of source code (\LaTeX) must retain the above
copyright notice, this list of conditions and the following disclaimer as the
first lines of this file unmodified.
\item Redistributions in compiled form (converted to
PDF, PostScript, RTF and other formats) must reproduce the above copyright
notice, this list of conditions and the following disclaimer in the
documentation and/or other materials provided with the distribution.
\end{enumerate}
THIS DOCUMENTATION IS PROVIDED BY THE LEDGERSMB PROJECT "AS IS" AND
ANY EXPRESS OR IMPLIED WARRANTIES, INCLUDING, BUT NOT LIMITED TO, THE IMPLIED
WARRANTIES OF MERCHANTABILITY AND FITNESS FOR A PARTICULAR PURPOSE ARE
DISCLAIMED. IN NO EVENT SHALL THE LEDGERSMB PROJECT BE LIABLE FOR
ANY DIRECT, INDIRECT, INCIDENTAL, SPECIAL, EXEMPLARY, OR CONSEQUENTIAL DAMAGES
(INCLUDING, BUT NOT LIMITED TO, PROCUREMENT OF SUBSTITUTE GOODS OR SERVICES;
LOSS OF USE, DATA, OR PROFITS; OR BUSINESS INTERRUPTION) HOWEVER CAUSED AND ON
ANY THEORY OF LIABILITY, WHETHER IN CONTRACT, STRICT LIABILITY, OR TORT
(INCLUDING NEGLIGENCE OR OTHERWISE) ARISING IN ANY WAY OUT OF THE USE OF THIS
DOCUMENTATION, EVEN IF ADVISED OF THE POSSIBILITY OF SUCH DAMAGE.

\tableofcontents{}

\listoffigures


\clearpage


\part{LedgerSMB and Business Processes}


\section{Introduction to LedgerSMB}

\subsection{What is LedgerSMB}

LedgerSMB is an open source financial accounting software program which is 
rapidly developing additional business management features.  Our goal is to 
provide a robust financial management suite for small to midsize busiensses.

\subsection{Why LedgerSMB}

\subsubsection{Advantages of LedgerSMB}

\begin{itemize}
\item Flexibility and Central Management 
\item Accessibility over the Internet (for some users) 
\item Relatively open data format
\item Integration with other tools 
\item Excellent accounting options for Linux users
\item Open Source 
\item Flexible, open framework that can be extended or modified to fit
your business. 
\item Security-conscious development community. 
\end{itemize}

\subsubsection{Key Features}

\begin{itemize}
\item Accounts Receivable 

\begin{itemize}
\item Track sales by customer 
\item Issue Invoices, Statements, Receipts, and more 
\item Do job costing and time entry for customer projects 
\item Manage sales orders and quotations 
\item Ship items from sales orders 
\end{itemize}
\item Accounts Payable 

\begin{itemize}
\item Track purchases and debts by vendor 
\item Issue RFQ's Purchase Orders, etc. 
\item Track items received from purchase orders 
\end{itemize}
\item Budgeting 

\begin{itemize}
\item Track expenditures and income across multiple departments 
\item Track all transactions across departments 
\end{itemize}
\item Check Printing 

\begin{itemize}
\item Customize template for any check form 
\end{itemize}
\item General Ledger 
\item Inventory Management 

\begin{itemize}
\item Track sales and orders of parts 
\item Track cost of goods sold using First In/First Out method 
\item List all parts below reorder point 
\item Track ordering requirements 
\item Track, ship, receive, and transfer parts to and from multiple warehouses 
\end{itemize}
\item Localization 

\begin{itemize}
\item Provide Localized Translations for Part Descriptions 
\item Provide Localized Templates for Invoices, Orders, Checks, and more 
\item Select language per customer, invoice, order, etc. 
\end{itemize}
\item Manufacturing 

\begin{itemize}
\item Track cost of goods sold for manufactured goods (assemblies) 
\item Create assemblies and stock assemblies, tracking materials on hand 
\end{itemize}
\item Multi-company/Multiuser 

\begin{itemize}
\item One isolated database per company 
\item Users can have localized systems independent of company data set 
\item Depending on configuration, users may be granted permission to different
      companies separately.
\end{itemize}
\item Point of Sale 

\begin{itemize}
\item Run multiple cash registers against main LedgerSMB installation 
\item Suitable for retail stores and more
\item Credit card processing via TrustCommerce 
\item Supports some POS hardware out of the box including:

\begin{itemize}
\item Logic Controls PD3000 pole displays (serial or parallel)
\item Basic text-based receipt printers
\item Keyboard wedge barcode scanners
\item Keyboard wedge magnetic card readers
\item Printer-attached cash drawers
\end{itemize}
\end{itemize}
\item Price Matrix 

\begin{itemize}
\item Track different prices for vendors and customers across the board 
\item Provide discounts to groups of customers per item or across the board 
\item Store vendors' prices independent of the other last cost in the parts
record 
\end{itemize}
\item Reporting 

\begin{itemize}
\item Supports all basic financial statements
\item Easily display customer history, sales data, and additional information 
\item Open framework allows for ODBC connections to be used to generate
reports using third party reporting tools. 
\end{itemize}
\item Tax 

\begin{itemize}
\item Supports Retail Sales Tax and Value Added Tax type systems 
\item Flexible framework allows one to customize reports to change the tax
reporting framework to meet any local requirement. 
\item Group customers and vendors by tax form for easy reporting (1099, EU VAT 
reporting, and similar)
\end{itemize}
\item Fixed Assets

\begin{itemize}
\item Group fixed assets for easy depreciation and disposal
\item Straight-line depreciation supported out of the box
\item Framework allows for easy development of production-based and time-based 
depreciation methods
\item Track full or partial disposal of assets
\end{itemize}
\end{itemize}

\subsection{Limitations of LedgerSMB}

\begin{itemize}
\item No payroll module (Payroll must be done manually) 
\item Some integration limitations 
\item Further development/maintenance requires a knowledge of a relatively
broad range of technologies 
\end{itemize}

\subsection{System Requirements of LedgerSMB}

\begin{itemize}
\item PostgreSQL 8.3 or higher
\item A CGI-enabled Web Server (for example, Apache) 
\item Perl 5.8.x or higher
\item An operating system which supports the above software (usually Linux,
though Windows, MacOS X, etc. do work) 
\item \LaTeX{}\ (optional) is required to create PDF or Postscript invoices 
\item The following CPAN modules:

\begin{itemize}
\item Data::Dumper
\item Log::Log4perl
\item Locale::Maketext
\item DateTime
\item Locale::Maketext::Lexicon
\item DBI
\item MIME::Base64
\item Digest::MD5
\item HTML::Entities
\item DBD::Pg
\item Math::BigFloat
\item IO::File
\item Encode
\item Locale::Country
\item Locale::Language
\item Time::Local
\item Cwd
\item Config::Std
\item MIME::Lite
\item Template
\item Error
\item CGI::Simple
\item File::MimeInfo
\end{itemize}

and these optional ones:
\begin{itemize}
\item Net::TCLink
\item Parse::RecDescent
\item Template::Plugin::Latex
\item XML::Twig
\item Excel::Template::Plus
\end{itemize}

\end{itemize}

\section{Installing LedgerSMB}

The process of installing LedgerSMB is described in detail
in the INSTALL file which comes with the distribution archive.
In the process is:
\begin{enumerate}
\item Install the base software: Web Server (Apache), Database server (PostgreSQL) and Perl
 from your distribution and package manager or source. Read the INSTALL file for details
\item Installing Perl module dependencies
 from your distribution and package manager or CPAN. Read the INSTALL file for details
\item Give the web server access to the ledgersmb directory
\item Edit ./ledgersmb/ledgersmb.conf to be able to access the database and locate the relevant PostgreSQL contrib scripts
\item Initializing a company database; database setup and upgrade script at http://localhost/ledgersmb/setup.pl
\item Login with your name (database username), password (database user password), Company (databasename)
\end{enumerate}


\section{User Account and Database Administration Basics}

LedgerSMB 1.3 offers a variety of tools for setting up new databases, and most
functionality (aside from creating new databases) is now offered dirctly within
the main applications.  LedgerSMB 1.2 users will note that the admin.pl script
is no longer included.

\subsection{Companies and Datasets}

LedgerSMB 1.3 stores data for each company in a separate "database".  A
database is a PostgreSQL concept for grouping tables, indexes, etc.

To create a database you will need to know a PostgreSQL superuser name and
password.  If you do not know this information you can set authentication to
"trust," then set the password, then set back to a more reasonable setting after
this process.  Plese see the PostgreSQL documentation for details.

Each company database must be named.  This name is essentially the system
identifier within PostgreSQL for the company's dataset.  The name for the
company database can only contain letters, digits and underscores.
Additionally, it must start with a letter.  Company database names are
case insensitive, meaning you can't create two separate company databases
called 'Ledgersmb' and 'ledgersmb'.

One way you can create databases fairly easily is by directing your web browser
to the setup.pl script at your installed ledgersmb directory.  So if the 
base URL is http://localhost/ledgersmb/, you can access the database setup and 
upgrade script at http://localhost/ledgersmb/setup.pl.  This is very different
from the approaches taken by LedgerSMB 1.2.x and earlier and SQL-Ledger, but
rather forms a wizard to walk you through the process.

An alternative method is the 'prepare-company-database.sh' script contributed by
Erik Huelsmann.  This script can be useful in creating and populating databases
from the command line and it offers a reference implementation written in BASH
for how this process is done.

The 'prepare-company-database.sh' script in the tools/ directory will set
up databases to be used for LedgerSMB. The script should be run as 'root'
because it wants to 'su' to the postgres user.  Alternatively, if you
know the password of the postgres user, you can run the script as any other
user.  You'll be prompted for the password.  Additionally, the script creates
a superuser to assign ownership of the created company database to. By
default this user is called 'ledgersmb'.  The reason for this choice is that
when removing the ledgersmb user, you'll be told about any unremoved parts
of the database, because the owner of an existing database can't be removed
until that database is itself removed.

The following invocation of the script sets up your first test company,
when invoked as the root user and from the root directory of the LedgerSMB
sources:

 \$ ./tools/prepare-company-database.sh --company testinc

The script assumes your PostgreSQL server runs on 'localhost' with
PostgreSQL's default port (5432).

Upon completion, it'll have created a company database with the name
'testinc', a user called 'ledgersmb' (password: 'LEDGERSMBINITIALPASSWORD'),
a single user called 'admin' (password: 'admin') and the roles required to
manage authorizations.

Additionally, it'll have loaded a minimal list of languages required
to succesfully navigate the various screens.

All these can be adjusted using arguments provided to the setup script. See
the output generated by the --help option for a full list of options.

\subsection{How to Create a User}

In the database setup workflow, a simple application user will be created.  This
user, by default, only has user management capabilities.  Ideally actual work 
should be done with accounts which have fewer permissions.

To set up a user, log in with you administrative credentials (created when
initializing the database) and then go to System/Admin Users/Add User.  From
here you can create a user and add appropriate permissions.

\subsection{Permissions}

Permissions in LedgerSMB 1.3 are enforced using database roles.  These are
functional categories and represent permissions levels needed to do basic tasks.
They are assigned when adding/editing users.

The roles follow a naming convention which allows several LSMB databases to
exist on the same PostgreSQL instance.  Each role is named lsmb\_\lbrack
dbname\rbrack\_\_ followed by the role name.  Note that two underscores separate
the database and role names.  If these are followed then the interface will
pick up on defined groups and display them along with other permissions.

\subsubsection{List of Roles}

Roles here are listed minus their prefix (lsmb\_$[$database name$]$\_\_, note 
the double underscore at the end of the prefix).

\begin{itemize}
\item Contact Management:  Customers and Vendors
      \begin{description}
      \item[contact\_read] Allows the user to read contact information
      \item[contact\_create] Allows the user to enter new contact information
      \item[contact\_edit] Allows the user to update the contact information
      \item[contact\_all] provides permission for all of the above.  Member of:
          \begin{itemize}
          \item contact\_read
          \item contact\_create
          \item contact\_edit
          \end{itemize}
      \end{description}
\item Batch Creation and Approval
      \begin{description}
      \item[batch\_create] Allows the user to create batches
      \item[batch\_post] Allows the user to take existing batches and post them
                         to the books
      \item[batch\_list] Allows the user to list batches and vouchers within
                         a batch.  This role also grants access to listing draft
                         tansactions (i.e. non-approved transactions not a 
                         part of a batch).  Member of:
            \begin{itemize}
            \item ar\_transaction\_list
            \item ap\_transaction\_list
            \end{itemize}
      \end{description}
\item AR:  Accounts Receivable
      \begin{description}
      \item[ar\_transaction\_create] Allows user to create transctions.  Member 
            of:
             \begin{itemize}
             \item contact\_read
             \end{itemize}
      \item[ar\_transaction\_create\_voucher].  Allows a user to create AR 
            transaction vouchers.  Member of:
            \begin{itemize}
            \item contact\_read
            \item batch\_create
            \end{itemize}
      \item[ar\_invoice\_create] Allows user to create sales invoices.  Member
            of:
            \begin{itemize}
            \item ar\_transaction\_create
            \end{itemize}
      \item[ar\_transaction\_list] Allows user to view transactions.  Member Of:
             \begin{itemize}
             \item contact\_read
             \end{itemize}
      \item[ar\_transaction\_all], all non-voucher permissions above, member of:
             \begin{itemize}
             \item ar\_transaction\_create
             \item ar\_transaction\_list
             \end{itemize}
      \item[sales\_order\_create] Allows user to create sales order.  Member of:
             \begin{itemize}
             \item contact\_read
             \end{itemize}
      \item[sales\_quotation\_create] Allows user to create sales quotations.
            Member of:
             \begin{itemize}
             \item contact\_read
             \end{itemize}
      \item [sales\_order\_list] Allows user to list sales orders. Member of:
             \begin{itemize}
             \item contact\_read
             \end{itemize}
      \item[sales\_quotation\_list] Allows a user to list sales quotations.
            Member of:
             \begin{itemize}
             \item contact\_read
             \end{itemize}
      \item[ar\_all]:  All AR permissions, member of:
             \begin{itemize}
             \item ar\_voucher\_all
             \item ar\_transaction\_all
             \item sales\_order\_create
             \item sales\_quotation\_create
             \item sales\_order\_list
             \item sales\_quotation\_list
             \end{itemize}
      \end{description}
\item AP:  Accounts Payable
      \begin{description}
      \item[ap\_transaction\_create] Allows user to create transctions.  Member 
            of:
             \begin{itemize}
             \item contact\_read
             \end{itemize}
      \item[ap\_transaction\_create\_voucher].  Allows a user to create AP
            transaction vouchers.  Member of:
            \begin{itemize}
            \item contact\_read
            \item batch\_create
            \end{itemize}
      \item[ap\_invoice\_create] Allows user to create vendor invoices.  Member
            of:
            \begin{itemize}
            \item ap\_transaction\_create
            \end{itemize}
      \item[ap\_transaction\_list] Allows user to view transactions.  Member Of:
             \begin{itemize}
             \item contact\_read
             \end{itemize}
      \item[ap\_transaction\_all], all non-voucher permissions above, member of:
             \begin{itemize}
             \item ap\_transaction\_create
             \item ap\_transaction\_list
             \end{itemize}
      \item[purchase\_order\_create] Allows user to create purchase orders,  
            Member of:
             \begin{itemize}
             \item contact\_read
             \end{itemize}
      \item[rfq\_create] Allows user to create requests for quotations.
            Member of:
             \begin{itemize}
             \item contact\_read
             \end{itemize}
      \item [purchase\_order\_list] Allows user to list purchase orders. 
             Member of:
             \begin{itemize}
             \item contact\_read
             \end{itemize}
      \item[rfq\_list] Allows a user to list requests for quotations.
            Member of:
             \begin{itemize}
             \item contact\_read
             \end{itemize}
      \item[ap\_all]:  All AP permissions, member of:
             \begin{itemize}
             \item ap\_voucher\_all
             \item ap\_transaction\_all
             \item purchase\_order\_create
             \item rfq\_create
             \item purchase\_order\_list
             \item rfq\_list
             \end{itemize}
      \end{description}
\item Point of Sale
      \begin{description}
      \item[pos\_enter] Allows user to enter point of sale transactions
          Member of:
         \begin{itemize}
         \item contact\_read
         \end{itemize}
      \item[close\_till] Allows a user to close his/her till
      \item[list\_all\_open] Allows the user to enter all open transactions
      \item[pos\_cashier] Standard Cashier Permissions.  Member of:
         \begin{itemize}
         \item pos\_enter
         \item close\_till
         \end{itemize}
      \item[pos\_all] Full POS permissions.  Member of:
         \begin{itemize}
         \item pos\_enter
         \item close\_till
         \item list\_all\_open
         \end{itemize}
      \end{description}
\item Cash Handling
      \begin{description}
      \item[reconciliation\_enter]  Allows the user to enter reconciliation 
             reports.
      \item[reconciliation\_approve] Allows the user to approve/commit
             reconciliation reports to the books.
      \item[reconciliation\_all]  Allows a user to enter and approve 
           reconciliation reports.  Don't use if separation of duties is 
           required.  Member of:
             \begin{itemize}
             \item reconciliation\_enter
             \item reconciliation\_approve
             \end{itemize}
      \item[payment\_process] Allows a user to enter payments.  Member of:
            \begin{itemize}
            \item ap\_transaction\_list
            \end{itemize}
      \item[receipt\_process] Allows a user to enter receipts.  Member of:
            \begin{itemize}
            \item ar\_transaction\_list
            \end{itemize}
      \item[cash\_all] All above cash roles.  Member of:
           \begin{itemize}
           \item reconciliation\_all
           \item payment\_process
           \item receipt\_process
           \end{itemize}
      \end{description}
\item Inventory Control
      \begin{description}
      \item[part\_create] Allows user to create new parts.
      \item[part\_edit] Allows user to edit parts
      \item[inventory\_reports] Allows user to run inventory reports
      \item[pricegroup\_create] Allows user to create pricegroups.
           Member of:
           \begin{itemize}
           \item contact\_read
           \end{itemize}
      \item[pricegroup\_edit]  Allows user to edit pricegroups
           Member of:
           \begin{itemize}
           \item contact\_read
           \end{itemize}
      \item[assembly\_stock]  Allows user to stock assemblies
      \item[inventory\_ship]  Allows user to ship inventory.  Member of:
           \begin{itemize}
           \item sales\_order\_list
           \end{itemize}
      \item[inventory\_receive] Allows user to receive inventory.  Member of:
           \begin{itemize}
           \item purchase\_order\_list
           \end{itemize}
      \item[inventory\_transfer] Allows user to transfer inventory between
           warehouses.
      \item[warehouse\_create] Allows user to create warehouses.
      \item[warehouse\_edit] Allows user to edit warehouses.
      \item[inventory\_all]  All permissions groups in this section.  
            Member of:
            \begin{itemize}
            \item part\_create
            \item part\_edit
            \item inventory\_reports
            \item pricegroup\_create
            \item pricegroup\_edit
            \item assembly\_stock
            \item inventory\_ship
            \item inventory\_transfer
            \item warehouse\_create
            \item warehouse\_edit
            \end{itemize} 
      \end{description}
\item GL:  General Ledger and General Journal 
      \begin{description}     
      \item[gl\_transaction\_create] Allows a user to create journal entries
           or drafts.
      \item[gl\_voucher\_create] Allows a user to create GL vouchers and 
           batches.
      \item[gl\_reports] Allows a user to run GL reports, listing all financial
           transactions in the database.  Member of:
           \begin{itemize}
           \item ar\_list\_transactions
           \item ap\_list\_transactions
           \end{itemize}
      \item[yearend\_run] Allows a user to run the year-end processes
      \item[gl\_all] All GL permissions.  Member of:
           \begin{itemize}
           \item gl\_transaction\_create
           \item gl\_voucher\_create
           \item gl\_reports
           \item yearend\_run
           \end{itemize} 
      \end{description}
\item Project Accounting
      \begin{description}
      \item[project\_create] Allows a user to create project entries.  User must
           have contact\_read permission to assing them to customers however.
      \item[project\_edit] Allows a user to edit a project.  User must
           have contact\_read permission to assing them to customers however.
      \item[project\_timecard\_add] Allows user to add time card.  Member of:
           \begin{itemize}
           \item contact\_read
           \end{itemize}
      \item[project\_timecard\_list] Allows a user to list timecards.  Necessary
            for order generation. Member of:
           \begin{itemize}
           \item contact\_read
           \end{itemize}
      \item[project\_order\_generate] Allows a user to generate orders from
           time cards.   Member of:
           \begin{itemize}
           \item project\_timecard\_list
           \item orders\_generate
           \end{itemize}
      \end{description}
\item Order Generation, Consolidation, and Management
      \begin{description}
      \item[orders\_generate] Allows a user to generate orders.  Member of:
           \begin{itemize}
           \item contact\_read
           \end{itemize}
      \item[orders\_sales\_to\_purchase]  Allows creation of purchase orders
           from sales orders.  Member of:
           \begin{itemize}
           \item orders\_generate
           \end{itemize}
      \item[orders\_purchase\_consolidate] Allows the user to consolidate 
           purchase orders.  Member of:
           \begin{itemize}
           \item orders\_generate
           \end{itemize}
      \item[orders\_sales\_consolidate]  Allows user to consolidate sales
           orders.  Member of:
           \begin{itemize}
           \item orders\_generate
           \end{itemize}
      \item[orders\_manage] Allows full management of orders.  Member of:
           \begin{itemize}
           \item project\_order\_generate
           \item orders\_sales\_to\_purchase
           \item orders\_purchase\_consolidate
           \item orders\_sales\_consolidate
           \end{itemize}
     \end{description}
\item Financial Reports
      \begin{description}
      \item[financial\_reports] Allows a user to run financial reports. 
           Member of:
           \begin{itemize}
           \item gl\_reports
           \end{itemize}
      \end{description}
\item  Batch Printing
      \begin{description}
      \item[print\_jobs\_list]  Allows the user to list print jobs.
      \item[print\_jobs] Allows user to print the jobs
           Member of: 
           \begin{itemize}
           \item print\_jobs\_list
           \end{itemize}
      \end{description}
\item System Administration
      \begin{description}
      \item[system\_settings\_list] Allows the user to list system settings.
      \item[system\_settings\_change] Allows user to change system settings.
           Member of:
           \begin{itemize}
           \item system\_settings\_list
           \end{itemize}
      \item[taxes\_set] Allows setting of tax rates and order.
      \item[account\_create]  Allows creation of accounts.
      \item[account\_edit]  Allows one to edit accounts.
      \item[auditor] Allows one to access audit trails.
      \item[audit\_trail\_maintenance] Allows one to truncate audit trails.
      \item[gifi\_create] Allows one to add GIFI entries.
      \item[gifi\_edit] Allows one to edit GIFI entries.
      \item[account\_all] A general group for accounts management.  Member of:
           \begin{itemize}
           \item account\_create
           \item account\_edit
           \item taxes\_set
           \item gifi\_create
           \item gifi\_edit
           \end{itemize}
      \item[department\_create] Allow the user to create departments.
      \item[department\_edit] Allows user to edit departments.
      \item[department\_all] Create/Edit departments.  Member of:
          \begin{itemize}
            \item department\_create
            \item department\_edit
          \end{itemize}
      \item[business\_type\_create] Allow the user to create business types.
      \item[business\_type\_edit] Allows user to edit business types.
      \item[business\_type\_all] Create/Edit business types.  Member of:
          \begin{itemize}
            \item business\_type\_create
            \item business\_type\_edit
          \end{itemize}
      \item[sic\_create] Allow the user to create SIC entries.
      \item[sic\_edit] Allows user to edit business types.
      \item[sic\_all] Create/Edit business types.  Member of:
          \begin{itemize}
            \item sic\_create
            \item sic\_edit
          \end{itemize}
      \item[tax\_form\_save] Allow the user to save the tax form entries.
      \item[template\_edit]  Allow the user to save new templates.  This 
            requires sufficient file system permissions.
      \item[users\_manage] Allows an admin to create, edit, or remove users.
            Member of:
            \begin{itemize}
            \item contact\_create
            \item contact\_edit
            \end{itemize}
      \item[system\_admin]  General role for accounting system administrators.
            Member of:
            \begin{itemize}
            \item system\_setting\_change
            \item account\_all
            \item department\_all
            \item business\_type\_all
            \item sic\_all
            \item tax\_form\_save
            \item template\_edit
            \item users\_manage
            \end{itemize}
      \end{description}
\item Manual Translation
      \begin{description}
      \item[language\_create] Allow user to create languages
      \item[language\_edit]  Allow user to update language entries
      \item[part\_translation\_create] Allow user to create translations of 
                                       parts to other languages.
      \item[project\_translation\_create] Allow user to create translations of
              project descriptions.
      \item[manual\_translation\_all] Full management of manual translations.
            Member of:
            \begin{itemize}
            \item language\_create
            \item language\_edit
            \item part\_translation\_create
            \item project\_translation\_create
            \end{itemize}
      \end{description}
\end{itemize}

\section{Contact Management}

Every business does business with other persons, corporate or natural.  They may
sell goods and services to these persons or they may purchase goods and 
services from these persons.  With a few exceptions those who are being sold
goods and services are tracked as customers, and those from whom goods and
services are being purchased from are vendors.  The actual formal distinction is
that vendors are entities that the business pays while customers pay the 
business.  Here are some key terms:

\begin{description}
\item[Credit Account] An agreement between your business and another person
or business and your business about the payment for the delivery of goods and
services on an ongoing basis.  These credit accounts define customer and vendor
relationships.
\item[Customer] Another person or business who pays your business money
\item[Vendor] Another person or business you pay money to.
\end{description}

Prior versions of LedgerSMB required that customers and vendors be entirely
separate.  In 1.3, however, a given entity can have multiple agreements with the
business, some being as a customer, and some being as a vendor.

All customers and vendors are currently tracked as companies, while employees 
are tracked as natural persons but this will be
changing in future versions so that natural persons can be tracked as customers
and vendors too.

Each contact must be attached to a country for tax reporting purposes.  Credit
accounts can then be attached to a tax form for that country (for 1099 reporting
in the US or EU VAT reporting).

\subsection{Addresses}
Each contact, whether an employee, customer, or vendor, can have one or more 
addresses attached, but only one can be a billing address.

\subsection{Contact Info}
Each contact can have any number of contact info records attached.  These convey
phone, email, instant messenger, etc. info for the individual.  New types of
records can be generated easily by adding them to the contact\_class table.

\subsection{Bank Accounts}
Each contact can have any number of bank accounts attached, but only one can be
the primary account for a given credit account.  There are only two fields here.
The first (BIC, or Banking Institution Code) represents the bank's identifier,
such as an ABA routing number, or a SWIFT code, while the second (IBAN)
represents the individual's account number.

\subsection{Notes}
In 1.3, any number of read-only notes can be attached either to an entity (in
which case they show up for all credit accounts for that entity), or a credit
account, in which case they show up only when the relevant credit account is 
selected.

\section{Chart of Accounts}

The Chart of Accounts provides a basic overview of the logical structure
of the accounting program. One can customize this chart to allow for
tracking of different sorts of information.

\subsection{Introduction to Double Entry Bookkeeping}
In order to set up your chart of accounts in LedgerSMB you will need to
understand a bit about double entry bookkeeping.  This section provides a
brief overview of the essential concepts.  There is a list of references
for further reading at the end.

\subsubsection{Business Entity}
You always want to keep your personal expenses and income separate from that of 
the business or you will not be able to tell how much money it is making (if
any).  For the same reason you will want to keep track of how much money
you put into and take out of the business so you will want to set up a
completely seperate set of records for it and treat it almost as if it had
a life of its own.

\subsubsection{Double Entry}
Examples:
\begin{itemize}
\item When you buy you pay money and receive goods.
\item When you sell you get money and give goods.
\item When you borrow you get money and give a promise to pay it back.
\item When you lend you give money and get a promise to pay it back.
\item When you sell on credit you give goods and get a promise to pay.
\item When you buy on credit you give a promise to pay and get goods.
\end{itemize}

You need to record both sides of each transaction: thus double entry.
Furthermore, you want to organize your entries, recording those having to
do with money in one place, value of goods bought and sold in another,
money owed in yet another, etc.  Hence you create accounts, and record each
half of each transaction in an appropriate account.  Of course, you won't
have to actually record the amount in more than one place yourself: the
program takes care of that.

\subsubsection{Accounts}

\begin{description}
\item[Assets] Money and anything that can be converted into money without 
reducing the net equity of the business.  Assets include money owed, money held,
goods available for sale, property, and the like.
\item[Liabilities] Debts owned by the business such as bank loans and unpaid 
bills. 
\item[Equity or Capital] What would be left for the owner if all the assets were 
converted to money and all the liabilities paid off ("Share Capital" on the
LedgerSMB default chart of accounts: not to be confused with "Capital Assets".)
\item[Revenue] Income from business activity: increases Equity  
\item[Expense] The light bill, the cost of goods sold, etc: decreases Equity
\end{description}

All other accounts are subdivisions of these.  The relationship between the
top-level accounts is often stated in the form of the Accounting Equation
(don't worry: you won't have to solve it):

Assets = Liabilities + Equity + (Revenue - Expense)

You won't actually use this equation while doing your bookkeeping, but it's
a useful tool for understanding how the system works.

\subsubsection{Debits and Credits}

The words "Debit" and "Credit" come from Latin roots. Debit is related to our 
word "debt" while credit can indicate a loan or something which edifies an 
individual or business.  The same applies to double entry accounting as it
involves equity accounts.  Debts debit equity, moneys owed to the business 
credit the business.  Credits to equity accounts make the business more valuable
while debits make it less.  Thus when you invest money in your business you are 
crediting that business (in terms of equity), and when you draw money, perhaps 
to pay yourself, you are debiting that business. 

Double entry accounting systems grew out of single entry ones.  The goal was to 
create a system which had inherent checks against human error.  Consequently
accounts and transactions are arranged such that debits across all accounts
always equal credit accounts.

If you invest money in your business that credits an equity account, but the 
other side of the transaction must thus be a debit.  Because the other side of
the transaction is an asset account (for example a bank account) it is debited.

Similarly as liability accounts increase, the equity of the business decreases.
Consequently, liabilities increase with credits.  Income and expense accounts
are often the flip sides of transactions involving assets and liailities and 
represent changes in equity.  Therefore they follow the same rules as equity 
accounts.

\begin{itemize}
\item Debits increase assets
\item Debits increase expense
\item Credits increase liabilities
\item Credits increase capital
\item Credits increase revenue
\end{itemize}

Examples:

You go to the bank and make a deposit.  The teller tells you that he is
going to credit your account.  This is correct: your account is money the
bank owes you and so is a liability from their point of view.  Your deposit
increased this liability and so they will credit it.  They will make an
equal debit to their cash account.  When you return you will debit your
bank deposits account because you have increased that asset and credit cash
on hand because you have decreased that one.


\subsubsection{Accrual}

Early accounting systems were usually run on a cash basis.  One generally did 
not consider money owed to affect the financial health of a company, so expenses
posted when paid as did income.

The problem with this approach is that it becomes very difficult or impossible 
to truly understand the exact nature of the financial health of a business.  One
cannot get the full picture of the financial health of a business because 
outstanding debts are not considered.  Futhermore, this does not allow for 
revenue to be tied to cost effectively, so it becomes difficult to assess how
profitable a given activity truly is.

To solve this problem, accrual-based systems were designed.  The basic principle
is that income and expense should be posted as they are incurred, or accrued.  
This allows one to track income relative to expense for specific projects or 
operations, and make better decisions about which activities will help one 
maximize profitability.

To show how these systems differ, imagine that you bill a customer for time and 
materials for a project you have just completed.  The customer pays the bill 
after 30 days.  In a cash based system, you would post the income at the time 
when the customer pays, while in an accrual system, the income is posted at the
time when the project is completed.

\subsubsection{Separation of Duties}
There are two important reasons not to trust accounting staff too much regarding
the quality of data that is entered into the system.  Human error does occur,
and a second set of eyes can help reduce that error considerably.

A second important reason to avoid trusting accounting staff too much is that
those with access to financial data are in a position to steal money from the
business.  All too often, this actually happens.  Separation of duties is the 
standard solution to this problem.

Separation of duties then refers to the process of separating parts of the
workflow such that one person's work must be reviewed and approved by someone
else.  For example, a book keeper might enter transactions and these might later
be reviewed by a company accountant or executive.  This thus cuts down both on
errors (the transaction is not on the books until it is approved), and on the
possibility of embezzlement.

Typically, the way duties are likely to be separated will depend on the specific
concerns of the company.  If fraud is the primary concern, all transactions will
be required to go through approval and nobody will ever be allowed to approve
their own transactions.  If fraud is not a concern, then typically transactions
will be entered, stored, and later reviewed/approved by someone different, but
allowances may be made allowing someone to review/approve the transactions
he/she entered.  This latter example doesn't strictly enforce separation of
duties, but encourages them nonetheless.

By default, LedgerSMB is set up not to strictly enforce the separation of
duties.  This can be changed by adding a database constraint to ensure that
batches and drafts cannot be approved by the same user that enters them.

In the age of computers, separation of duties finds one more important
application:  it allows review and approval by a human being of automatically
entered transactions.  This allows the accounting department to double-check
numbers before they are posted to the books and thus avoid posting incorrect
numbers (for example, due to a software bug in custom code).

Unapproved transactions may be deleted as they are not full-fledged transactions
yet.  Approved transactions should be reversed rather than deleted.

Separation of duties is not available yet for sales/vendor invoice documents,
but something similar can be handled by feeding them through the order entry
workflow (see \ref{oe}).

\subsubsection{References}

\url{http://www.accounting-and-bookkeeping-tips.com/learning-accounting/accounting-basics-credit.htm}\\
Discussion of debits and credits as well as links to other accounting subjects.\\

\noindent \url{http://www.computer-consulting.com/accttips.htm}\\
Discussion of double entry bookkeeping.\\

\noindent \url{http://www.minnesota.com/~tom/sql-ledger/howtos/}\\
A short glossary, some links, and a FAQ (which makes the "credit=negative
number" error).  The FAQ focuses on SQL-Ledger, LedgerSMB's ancestor.\\

\noindent \url{http://bitscafe.com/pub2/etp/sql-ledger-notes\#expenses}\\
Some notes on using SQL-Ledger (LedgerSMB's ancestor).\\

\noindent \url{http://en.wikipedia.org/wiki/List\_of\_accounting\_topics}\\
Wikipedia articles on accounting.\\

\noindent \url{http://www.bized.ac.uk/learn/accounting/financial/index.htm}\\
Basic accounting tutorial.\\

\noindent \url{http://www.asset-analysis.com/glossary/glo\_index.html}\\
Financial dictionary and glossary.\\

\noindent \url{http://www.geocities.com/chapleaucree/educational/FinanceHandbook.html}\\
Financial glossary.\\

\noindent \url{http://www.quickmba.com/accounting/fin/}\\
Explanation of fundamentals of accounting, including good discussions
of debits and credits and of double-entry.


\subsection{General Guidelines on Numbering Accounts}

In general, most drop-down boxes in LedgerSMB order the accounts
by account number. Therefore by setting appropriate account numbers,
one can affect the default values.

A second consideration is to try to keep things under each heading
appropriate to that heading. Thus setting an account number for a
bank loan account in the assets category is not generally advisable.

If in doubt, review a number of bundled chart of accounts templates to see what 
sorts of numbering schemes are used.

\subsection{Adding/Modifying Accounts}

These features are listed under System-\textgreater Chart of Accounts.
One can list the accounts and click on the account number to modify
them or click on the \char`\"{}add account\char`\"{} option to create
new accounts.

\begin{itemize}
\item Headings are just broad categories and do not store values themselves,
while accounts are used to store the transactional information. 
\item One cannot have an account that is  a summary account (like AR)
and also has another function. 
\item GIFI is mostly of interest to Canadian customers but it can be used
to create reports of account hierarchies. 
\end{itemize}

\subsection{Listing Account Balances and Transactions}

One can list the account balances via the Reports-\textgreater Chart
of Accounts report. Clicking on the account number will provide a
ledger for that account.


\section{Administration}

This section covers other (non-Chart of Accounts) aspects to the
setup of the LedgerSMB accounting package. These are generally accessed
in the System submenu.


\subsection{Taxes, Defaults, and Preferences}
Since LedgerSMB 1.2, sales tax has been modular, allowing for different tax 
accounts to be goverend by different rules for calculating taxes (although only 
one such module is supplied with LedgerSMB to date).  This allows one to 
install different tax modules and then 
select which taxes are applied by which programming modules.  The sales tax 
module has access to everything on the submitted form so it is able to make
complex determinations on what is taxable based on arbitrary criteria.

The tax rules drop-down box allows one to select any installed tax module 
(LedgerSMB 1.3 ships only with the simple module), while the ordering is an
integer which allows one to specify a tax run which occurs on the form after
any rules with lower entries in this box.  This allows for compounding of sales tax (for example, when PST applies to the total and GST as well).

\subsubsection{Adding A Sales Tax Account}

Sales Tax is collected on behalf of a state or national government
by the individual store. Thus a sales tax account is a liability--
it represents money owed by the business to the government.

To add a sales tax account, create an account in the Chart of Accounts
as a liability account, check all of the ``tax'' checkboxes.

Once this account is created, one can set the tax amount.


\subsubsection{Setting a Sales Tax Amount}

Go to System-\textgreater Defaults and the tax account will be listed
near the bottom of the page. The rate can be set there.


\subsubsection{Default Account Setup}

These accounts are the default accounts for part creation and foreign
exchange tracking.


\subsubsection{Currency Setup}

The US accounts list this as USD:CAD:EUR. One can add other currencies
in here, such as IDR (Indonesian Rupiah), etc. Currencies are separated
by colons.


\subsubsection{Sequence Settings}

These sequences are used to generate user identifiers for quotations,
invoices, and the like. If an identifier is not added, the next number
will be used.

A common application is to set invoices, etc. to start at 1000 in
order to hide the number of issued invoices from a customer.

Leading zeros are preserved.  Other special values which can be embedded using
$<$?lsmb ?$>$ tags include:

\begin{description}
\item[DATE] expands to the current date
\item[YYMMDD] expands to a six-digit version of the date.  The components of 
this date can be re-arranged in any order, so MMDDYY, DDMMYY, 
or even just MMYY are all options.
\item[NAME] expands to the name of the customer or vendor
\item[BUSINESS] expands to the type of business assigned to the customer or 
ventor.
\item[DESCRIPTION] expands to the description of the part.  Valid only for parts.
\item[ITEM] expands to the item field.  Valid only for parts.
\item[PERISCOPE] expands to the partsgroup.  Valid only for parts.
\item[PHONE] expands to the telephone number for customers and vendors.
\end{description}

\subsection{Audit Control}

Auditability is a core concern of the architects of any accounting
system. Such ensures that any modification to the accounting information
leaves a trail which can be followed to determine the nature of the
change. Audits can help ensure that the data in the accounting system
is meaningful and accurate, and that no foul play (such as embezzlement)
is occurring.


\subsubsection{Explaining transaction reversal}

In paper accounting systems, it was necessary to have a means to authoritatively
track corrections of mistakes. The means by which this was done was
known as ``transaction reversal.''

When a mistake would be made, one would then reverse the transaction
and then enter it in correctly. For example, let us say that an office
was renting space for \$300 per month. Let us say that they inadvertently
entered it in as a \$200 expense.

The original transaction would be:

\begin{tabular}{l|r|r}
Account &
Debit &
Credit \tabularnewline
\hline 
5760 Rent &
\$200 &
\tabularnewline
2100 Accounts Payable &
&
\$200\tabularnewline
\end{tabular}

The reversal would be:

\begin{tabular}{l|r|r}
Account &
Debit &
Credit \tabularnewline
\hline 
5760 Rent &
&
\$200\tabularnewline
2100 Accounts Payable &
\$200 &
\tabularnewline
\end{tabular}

This would be followed by re-entering the rent data with the correct
numbers. This was meant to ensure that one did not erase data from
the accounting books (and as such that erasing data would be a sign
of foul play).

LedgerSMB has a capability to require such reversals if the business
deems this to be necessary. When this option is enabled, existing
transactions cannot be modified and one will need to post reversing
transactions to void existing transactions before posting corrected
ones.

Most accountants prefer this means to other audit trails because it
is well proven and understood by them.


\subsubsection{Close books option}

You cannot alter a transaction that was entered before the closing date.  


\subsubsection{Audit Trails}

This option stores additional information in the database to help
auditors trace individual transactions. The information stored, however,
is limited and it is intended to be supplemental to other auditing
facilities.

The information added includes which table stored the record, which
employee entered the information, which form was used, and what the
action was. No direct financial information is included.


\subsection{Departments}

Departments are logical divisions of a business. They allow for budgets
to be prepared for the individual department as well as the business
as a whole. This allows larger businesses to use LedgerSMB to meet
their needs.


\subsubsection{Cost v Profit Centers.}

In general business units are divided into cost and profit centers.
Cost centers are generally regarded as business units where the business
expects to lose money and profit centers are where they expect to
gain money. For example, the legal department in most companies is
a cost center.

One of the serious misunderstandings people run up against is that
LedgerSMB tends to more narrowly define cost and profit centers than
most businesses do. In LedgerSMB a cost center is any department
of the business that does not issue AR transactions. Although many
businesses may have cost centers (like technical support) where customer
fees may subsidize the cost of providing the service, in LedgerSMB,
these are profit centers.

LedgerSMB will not allow cost centers to be associated with AR transactions.
So if you want this functionality, you must create the department
as a profit center.


\subsection{Warehouses}

LedgerSMB has the ability to track inventory by warehouse. Inventory
items can be moved between warehouses, and shipped from any warehouse
where the item is in stock. We will explore this concept more later.


\subsection{Languages}

Languages allow for goods and services to be translated so that one
can maintain offices in different countries and allow for different
goods and service descriptions to be translated to different languages
for localization purposes.


\subsection{Types of Businesses}

One can create types of businesses and then give them discounts across
the board. For example, one might give a firm that uses one's services
as a subcontractor a 10\% discount or more.


\subsection{Misc.}


\subsubsection{GIFI}

GIFI is a requirement for Canadian customers. This feature allows
one to link accounts with Canadian tax codes to simplify the reporting
process. Some European countries now use a similar system.

People that don't otherwise have a use for GIFI can use it to create reports
which agregate accounts together.

\subsubsection{SIC}

Standard Industrial Classification is a way of tracking the type of
business that a vendor or customer is in. For example, an accountant
would have an SIC of 8721 while a graphic design firm would have an
SIC of 7336. The classification is hierarchical so one could use this
field for custom reporting and marketing purposes.


\subsubsection{Overview of Template Editing}

The templates for invoices, orders, and the like can be edited from
within LedgerSMB. The submenus within the System submenu such as
HTML Templates, Text Templates and \LaTeX{} templates provide access
to this functionality.


\subsubsection{Year-end}

Although the Year-end functionality in LedgerSMB is very useful,
it does not entirely make the process simple and painless. One must
still manually enter adjustments prior to closing the books. The extent
to which these adjustments are necessary for any given business is
a matter best discussed with an accountant.

The standard way books are normally closed at the end of the year
is by moving all adjusted\footnote{Adjustments would be entered via the General 
Ledger. The exact process is beyond the scope of this document, however.} income
and expenses to an equity account usually called `Retained
Earnings.' Assets and liabilities are not moved. Equity drawing/dividend
accounts are also moved, but the investment accounts are not. The
reasoning behind this process is that one wants a permanent record
of the amount invested in a business, but any dividends ought not
to count against their recipients when new investors are brought on
board.

LedgerSMB automatically moves all income and expense into the specified
year-end/retained earnings account. It does not move the drawing account,
and this must be done manually, nor does it automate the process of
making adjustments.

Contrary to its name, this function can close the books at any time,
though this would likely be of limited use.

Once the books are closed, no transactions can be entered into the closed
period.  Additionally the year end routines cannot be run if there are
unapproved transactions in a period to be closed.

\subsection{Options in the ledger-smb.conf}

The ledger-smb.conf configures the software by assigning site-wide
variables. Most of these should be left alone unless one knows what
one is doing. However, on some systems some options might need to
be changed, so all options are presented here for reference:

\begin{description}
\item[auth] is the form of authenticaiton used.  If in doubt use `DB' auth.
\item[decimal\_places] Number of decimal places for money.
\item[templates] is the directory where the templates are stored. 
\item[sendmail] is the command to use to send a message. It must read the
email from standard input. 
\item[language] allows one to set the language for the login screen and
admin page. 
\item[latex] tells LedgerSMB whether \LaTeX{} is installed. \LaTeX{} is
required for generating Postscript and PDF invoices and the like. 
\item[Environmental variables] can be set here too. One
can add paths for searching for \LaTeX{}, etc. 
\item[Printers] section can be used to set a hash table of printers for the software.
The primary example is\\
$[$printers$]$\\
Default = lpr\\
Color = lpr -PEpson \\%
 However, this can use any program that can accept print documents
(in Postscript) from standard input, so there are many more possibilities. 
\item[database] provides connection parameters for the database, typically the
host and port, but also the location of the contrib scripts (needed for the
setup.pl), and the default namespace.
\end{description}

\section{Goods and Services}

The Goods and Services module will focus on the definition of goods
and services and the related accounting concepts.


\subsection{Basic Terms}

\begin{description}
\item [{COGS}] is Cost of Goods Sold. When an item is sold, then the expense
of its purchase is accrued as attached to the income of the sale.
\item [{List}] Price is the recommended retail price. 
\item [{Markup}] is the percentage increase that is applied to the last
cost to get the sell price. 
\item [{ROP}] is re-order point. Items with fewer in stock than this will
show up on short reports. 
\item [{Sell}] Price is the price at which the item is sold. 
\end{description}

\subsection{The Price Matrix}

It is possible to set different prices for different groups of customers,
or for different customers individually. Similarly, one can track
different prices from different vendors along with the required lead
time for an order.


\subsection{Pricegroups}

Pricegroups are used to help determine the discount a given customer
may have.


\subsection{Groups}

Groups represent a way of categorizing POS items for a touchscreen
environment. It is not fully functional yet, but is sufficient that
with some stylesheet changes, it could be made to work.


\subsection{Labor/Overhead}

Labor/overhead is usually used for tracking manufacturing expenses.
It is not directly billed to a customer. It is associated with an
expense/Cost of Goods Sold (COGS) account.


\subsection{Services}

Services include any labor that is billed directly to the customer.
It is associated with an expense/COGS account and an income account.
Services can be associated with sales tax.


\subsubsection{Shipping and Handling as a Service}

One approach to dealing with shipping and handling is to add it as
a service. Create a service called ``Shipping and Handling,''
with a sell price \$1 per unit, and a 0\% markup. Bill it as \$1 per
unit. This allows one to add the exact amount of shipping and handling
as necessary.


\subsection{Parts}

A part is any single item you might purchase and either might resell
or use in manufacturing an assembly. It is linked to an expense/COGS
account, an income account, and an inventory account. Parts can be
associated with sales tax.


\subsection{Assemblies and Manufacturing}

Manufacturers order parts but they sell the products of their efforts.
LedgerSMB supports manufacturing using the concept of assemblies.
An assembly is any product which is manufactured on site. It consists
of a selection of parts, services, and/or labor and overhead. Assemblies
are treated as parts in most other regards.

However, one cannot order assemblies from vendors. One must instead
order the components and stock them once they are manufactured.


\subsubsection{Stocking Assemblies}

One stocks assemblies in the Stock Assembly entry on the Goods and
Services submenu. When an assembly is stocked the inventory is adjusted
properly.

The Check Inventory option will cause LedgerSMB to refuse to stock
an assembly if the inventory required to produce the assembly would
drop the part below the reorder point.


\subsection{Reporting}


\subsubsection{All Items and Parts Reports}

The All Items report provides a unified view of assemblies, parts, services,
and labor for the company, while the Parts report confines it to parts.

Types of reports are: 

\begin{description}
\item [{Active}] lists all items not marked as obsolete. 
\item [{On}] Hand lists current inventory .
\item [{Short}] Lists all items which are stocked below their ROP.
\item [{Obsolete}] Lists all items which are marked as obsolete.
\item [{Orphaned}] Lists all items which have never had a transaction associated
with them. 
\end{description}
One can also list these goods by invoice, order, or quotation.

For best results, it is a good idea to enter some AR and AP data before
running these reports.


\subsubsection{Requirements}

This report is designed to assist managers determine the quantities
of goods to order and/or stock. It compares the quantity on hand with
the activity in a given time frame and provides a list of goods which
need to be ordered and the relevant quantity.


\subsubsection{Services and Labor}

This is similar to the Parts and All Items menu but only supports
Active, Obsolete, and Orphaned reports.


\subsubsection{Assemblies}

This is similar to the Parts and All Items reports but it also provides
an ability to list individual items in the assemblies as well.

AP Invoices, Purchase Orders, and RFQ's are not available on this
report.


\subsubsection{Groups and Pricegroups}

These reports provide a simple interface for locating groups and pricegroups.
The report types are similar to what they are for services.


\subsection{Translations}

One can add translations so that they show up in the customer's native
language in the issued invoice.

To issue translations, one must have languages defined. One can then
add translations to descriptions and part groups.


\subsection{How Cost of Goods Sold is tracked}

Cost of Goods Sold is tracked on a First-In, First-out (FIFO) basis.
When a part is purchased, its cost is recorded in the database. The
cost of the item is then added to the inventory asset account. When
the good is sold, the cost of the item is moved to the cost of goods
sold account.

This means that one must actually provide invoices for all goods entered
at their actual cost. If one enters in \$0 for the cost, the cost
of goods sold will also be \$0 when the item is sold. We will cover
this entire process in more depth after we cover the AP and AR units
below.

\section{Transaction Approval}

With the exception of Sales/Vendor Invoices (with inventory control), any
financial transaction entered by default must be approved before it shows up in
financial reports.  Because there are two ways these can be set up, there are
two possibly relevant workflows.

For sales/vendor invoices where goods and services are tracked (as distinct
from AR/AP transactions which only track amounts), the separation of duties
interface is not complete.  Here, you should use orders for initial entry, and
convert these to invoices.

\subsection{Batches and Vouchers}
Often larger businesses may need to enter batches of transactions which may need
to be approved or rolled back as a batch.  Batches are thus a generic
"container" for vouchers.  A given batch can have AR, AP, payment, receipt, and
GL vouchers in it together.  That same batch, however, will be classified by its
main purpose.

For example, one may have a batch for processing payments.  That batch may
include payment transactions, but also related ar/ap transactions (relating to
specific charges relating to payment or receipt).  The batch would still be
classified as a payment batch however.

In the ``Transaction Approval/Batches'' screen, one can enter search criteria
for batches. 

The next screen shows a list of batches including control codes, amounts covered
in the batch, and descriptions.  Clicking on the control code leads you to a
details screen where specific vouchers can be dropped from the batch.

When the batch is approved, all transactions in it are approved with it.

\subsection{Drafts}

Drafts are single transactions which have not yet been approved.   For example,
a journal entry or AR Transaction would become a ``draft'' that would need to be
approved after entry.

As with batches, one searches for drafts on the first screen and then can
approve either on the summary screen or the details screen (by clicking
through).

\section{AP}


\subsection{Basic AP Concepts}

The Accounts Payable module tracks all financial commitments that
the company makes to other businesses. This includes rent, utilities,
etc. as well as orders of goods and services.


\subsection{Vendors}

A vendor is any business that the company agrees to pay money to.

One can enter vendor information under AP-\textgreater Vendors-\textgreater
Add Vendor. The vendor list can be searched under AP-\textgreater
Vendors-\textgreater Reports-\textgreater Search.

Please see the Contact Management section above for more on manageing vendors.

\subsection{AP Transactions}

AP Transactions are generally used for items other than goods and
services. Utilities, rent, travel expenses, etc. could be entered
in as an AP transaction.

If the item is paid partially or in full when the transaction is entered,
one can add payments to the payment section.

All other payments can and should be entered under cash payment (below).

The PO Number and Order Number fields are generally used to track
associations with purchase orders sent to vendors, etc. These fields
can be helpful for adding misc. expenses to orders for reporting purposes.

The department drop-down box appears when one has created one or more
departments. A transaction is not required to be associated with a
department, but one can use this feature for budget tracking.

With AP Transactions, there is no option for internal notes. All notes
will appear on any printed version of the transaction.

Note: Printing a transaction does not post it. No data is committed
until the invoice is posted.


\subsection{AP Invoices}

AP Invoices are used to enter in the receipt of goods and services.
Goods and services are deemed entered into the inventory when they
are invoiced.

This screen is reasonably similar to the AP Transaction Screen, though
the part entry section is a bit different.

The AP Invoice section has a capacity to separate internal notes from
notes printed on the invoice. Note, however, that since these are
received invoices, it is rare that one needs this ability.

Note that LedgerSMB can search for partial part numbers or descriptions.

Also if you have a group you can use this to select the part.

To remove a line item from an invoice or order, delete the partnumber
and click update.


\subsubsection{Correcting an AP Invoice}

If an invoice is entered improperly, the methods used to correct it
will vary depending on whether transaction reversal is enforced or
not. If transaction reversal is not enforced, one can simply correct
the invoice or transaction and repost. Note, however, that this violates 
generally accepted accounting principles.

If transaction reversal is in effect, one needs to create a duplicate 
invoice with exactly opposite values entered. If one part was listed as 
received, then one should enter a negative one for the quantity. Then one 
can enter the invoice number as the same as the old one. Add an R to the 
end to show that it is a reversing transaction. Once this is posted, one can 
enter the invoice correctly.


\subsection{Cash payment And Check Printing}

It is a bad idea to repost invoices/transactions just to enter a payment. 
The Cash-\textgreater Payment window allows one to enter payments against 
AP invoices or transactions.

The printing capability can be used to print checks. The default template
is NEBS 9085, though you can use 9082 as well (as Quickbooks does).

The source field is used to store an identifying number of the source
document, such as the check number. One must select the item to have
it paid, and then enter the amount. One can then print a check.


\subsubsection{Batch Payment Entry Screen}

For bulk payment entry, we provide the batch payment workflow.  You can use this
to pay any or all vendors filtered by business class or the like.  Each payment
batch is saved, and hits the books after it is reviewed.  It is possible to pay
over ten thousand invoices a week using this interface.  It is found under
Cash/Vouchers/Payments.

\subsection{Transaction/Invoice Reporting}


\subsubsection{Transactions Report}

This report is designed to help you locate AP transactions based on
various criteria. One can search by vendor, invoice number, department,
and the like. One can even search by the shipping method.

The summary button will show what was placed where, while the details
button will show all debits and credits associated with the transaction.

To view the invoice, click on the invoice number. In the detail view,
to view the account transactions as a whole, click on the account
number.

Open invoices are ones not fully paid off, while closed invoices
are those that have been paid.


\subsubsection{Outstanding Report}

The outstanding report is designed to help you locate AP transactions
that are not paid yet. The ID field is mostly useful for locating
the specific database record if a duplicate invoice number exists.


\subsubsection{AP Aging Report}

This report can tell you how many invoices are past due and by how
much.

A summary report just shows vendors while a detail report shows individual
invoices.


\subsubsection{Tax Paid and Non-taxable Report}

These reports have known issues.  It is better to use the GL reports and filter 
accordingly.

\subsection{Vendor Reporting}


\subsubsection{Vendor Search}

The Vendor Search screen can be used to locate vendors or AP transactions
associated with those vendors.

The basic types of reports are:

\begin{description}
\item [{All}] Lists all vendors 
\item [{Active}] Lists those vendors currently active 
\item [{Inactive}] Lists those vendors who are currently inactive. time
frame. 
\item [{Orphaned}] Lists those vendors who do not have transactions associated
with them. These vendors can be deleted. 
\end{description}
One can include purchase orders, Requests for Quotations, AP invoices,
and AP transactions on this report as well if they occur between the
from and to dates.


\subsubsection{Vendor History}

This report can be used to obtain information about the past goods
and services ordered or received from vendors. One can find quantities,
partnumber, and sell prices on this report. This facility can be used
to search RFQ's, Purchase Orders, and AP Invoices.


\section{AR}


\subsection{Customers}

Customers are entered in using the AR-\textgreater Customers-\textgreater
Add Customer menu.

The salesperson is autopopulated with the current user who is logged
in. Otherwise, it looks fairly similar to the Vendor input screen.
Customers, like vendors can be assigned languages, but it is more
important to do so because invoices will be printed and sent to them.

The credit limit field can be used to assign an amount that one is
willing to do for a customer on credit.


\subsubsection{Customer Price Matrix}

The price list button can be used to enter specific discounts to the
customer, and groups of customers can be assigned a pricegroup for
the purpose of offering specific discounts on specific parts to the
customer. Such discounts can be temporary or permanent.


\subsection{AR Transactions}

AR Transactions are where one can add moneys owed the business by
customers. One can associate these transactions with income accounts,
and add payments if the item is paid when the invoice is issued.

The PO number field is used to track the PO that the customer sent.
This makes it easier to find items when a customer is asking for clarification
on a bill, for example.


\subsection{AR Invoices}

AR Invoices are designed to provide for the delivery of goods and
services to customers. One would normally issue these invoices at
the time when the everything has been done that is necessary to get
paid by the customer.

As with AP invoices, one can search for matches to partial part numbers
and descriptions, and enter initial payments at this screen.


\subsection{Cash Receipt}

The Cash-\textgreater Receipt screen allows you to accept prepayments
from customers or pay single or multiple invoices after they have
been posted. One can print a receipt, however the current templates
seem to be based on check printing templates and so are unsuitable
for this purpose. This presents a great opportunity for improvement.


\subsubsection{Cash Receipts for multiple customers}

The cash-\textgreater receipts screen allows you to accept payments
on all open customer invoices of all customers at once. One could
print (directly to a printer only) all receipts to be sent out if
this was desired.


\subsection{AR Transaction Reporting}

The AR Outstanding report is almost identical to the AP Outstanding
report and is not covered in any detail in this document.


\subsubsection{AR Transactions Report}

This is almost identical to the AP Transactions Report.

If a customer's PO has been associated with this transaction, one
can search under this field as well.


\subsubsection{AR Aging Report}

This report is almost identical to the AP Aging report, with the exception
that one can print up statements for customer accounts that are overdue.
One more application is to calculate interest based on balance owed
so that these can be entered as AR transactions associated with the
customer.


\subsection{Customer Reporting}

These reports are almost identical to the AP Vendor reports and are
not discussed in these notes.


\section{Projects}


\subsection{Project Basics}

A project is a logical collection of AR and AP transactions, orders,
and the like that allow one to better manage specific service or product
offerings. LedgerSMB does not offer comprehensive project management
capabilities, and projects are only used here as they relate to accounting.

One can also add translated descriptions to the project names as well.


\subsection{Timecards}

Timecards allow one to track time entered on specific services. These
can then be used to generate invoices for the time entered.

The non-chargeable is the number of hours that are not billed on the
invoice.

One can then generate invoices based on this information.

The project field is not optional.


\subsection{Projects and Invoices}

One can select the project id for line items of both AR and AP invoices.
These will then be tracked against the project itself.


\subsection{Reporting}


\subsubsection{Timecard Reporting}

The Timecard Report allows one to search for timecards associated
with one or more projects. One can then use the total time in issuing
invoices (this is not automated yet).


\subsubsection{Project Transaction Reporting}

The Standard or GIFI options can be used to create different reports
(for example, for Canadian Tax reporting purposes).

This report brings up a summary that looks sort of like a chart of
accounts. Of one clicks on the account numbers, one can see the transactions
associated with the project.


\subsubsection{List of Projects}

This provides a simple way of searching for projects to edit or modify.


\subsection{Possibilities for Using Projects}

\begin{itemize}
\item One can use them similar to departments for tracking work done for
a variety of customers. 
\item One can use them for customer-specific projects. 
\end{itemize}

\section{Quotations and Order Management}
\label{oe}

This unit will introduce the business processes that LedgerSMB allows.
These processes are designed to allow various types of businesses
to manage their orders and allow for rudimentary customer relationship
management processes to be built around this software. In this section,
we will introduce the work flow options that many businesses may use
in their day-to-day use of the software.


\subsection{Sales Orders}

Sales orders represent orders from customers that have not been delivered
or shipped yet. These orders can be for work in the future, for
back ordered products, or work in progress. A sales order can be generated
form an AR invoice or from a quotation automatically.


\subsection{Quotations}

Quotations are offers made to a customer but to which the customer
has not committed to the work. Quotations can be created from Sales
orders or AR Invoice automatically.


\subsection{Shipping}

The Shipping module (Shipping-\textgreater Shipping) allows one to
ship portions or entireties of existing sales orders, printing pick
lists and packing slips.

One can then generate invoices for those parts that were shipped.

In general, one will be more likely to use these features if they
have multiple warehouses that they ship from. More likely most customers
will just generate invoices from orders.


\subsection{AR Work Flow}


\subsubsection{Service Example}

A customer contacts your firm and asks for a quote on some services.
Your company would create a quotation for the job and email it to
the customer or print it and mail it. Once the customer agrees to
pay, one creates a sales order from the quotation.

When the work is completed, the sales order is converted into a sales 
invoice and this is presented to the customer as a bill.

Note that in some cases, this procedure may be shortened. If the customer
places an order without asking for a quotation and is offered a verbal
quote, then one might merely prepare the sales order.

%
\begin{figure}[hbtp]
 


\caption{Simple AR Service Invoice Workflow Example}

\setlength{\unitlength}{3947sp}%
%
\begingroup\makeatletter\ifx\SetFigFont\undefined%
\gdef\SetFigFont#1#2#3#4#5{%
  \reset@font\fontsize{#1}{#2pt}%
  \fontfamily{#3}\fontseries{#4}\fontshape{#5}%
  \selectfont}%
\fi\endgroup%
\begin{picture}(2674,4374)(5939,-4273)
\thinlines
{\color[rgb]{0,0,0}\put(7276,-736){\vector( 0,-1){1050}}
}%
{\color[rgb]{0,0,0}\put(7276,-2536){\vector( 0,-1){825}}
}%
{\color[rgb]{0,0,0}\put(5951,-736){\framebox(2650,825){}}
}%
{\color[rgb]{0,0,0}\put(5951,-2536){\framebox(2650,750){}}
}%
{\color[rgb]{0,0,0}\put(5951,-4261){\framebox(2650,900){}}
}%
\put(7201,-2161){\makebox(0,0)[b]{\smash{\SetFigFont{12}{14.4}{\rmdefault}{\mddefault}{\updefault}{\color[rgb]{0,0,0}Sales Order}%
}}}
\put(7201,-3811){\makebox(0,0)[b]{\smash{\SetFigFont{12}{14.4}{\rmdefault}{\mddefault}{\updefault}{\color[rgb]{0,0,0}AR Invoice}%
}}}
\put(7276,-361){\makebox(0,0)[b]{\smash{\SetFigFont{12}{14.4}{\rmdefault}{\mddefault}{\updefault}{\color[rgb]{0,0,0}Quotation}%
}}}
\end{picture}
 
\end{figure}



\subsubsection{Single Warehouse Example}

A customer contacts your firm and asks for a quotation for shipping
a part. You would create the quotation and when you get confirmation,
convert it to an order. Once the parts are in place you could go to
shipping and ship the part.

The billing department can then generate the invoice from the sales
order based on what merchandise has been shipped and mail it to the
customer.

Note that this requires that you have the part in your inventory.

%
\begin{figure}[hbtp]
 


\caption{AR Workflow with Shipping}

\setlength{\unitlength}{3947sp}%
%
\begingroup\makeatletter\ifx\SetFigFont\undefined%
\gdef\SetFigFont#1#2#3#4#5{%
  \reset@font\fontsize{#1}{#2pt}%
  \fontfamily{#3}\fontseries{#4}\fontshape{#5}%
  \selectfont}%
\fi\endgroup%
\begin{picture}(2424,6024)(4789,-6073)
\thinlines
{\color[rgb]{0,0,0}\put(4801,-886){\framebox(2400,825){}}
}%
{\color[rgb]{0,0,0}\put(4801,-2536){\framebox(2400,825){}}
}%
{\color[rgb]{0,0,0}\put(4801,-4261){\framebox(2400,825){}}
}%
{\color[rgb]{0,0,0}\put(4801,-6061){\framebox(2400,900){}}
}%
{\color[rgb]{0,0,0}\put(6001,-886){\vector( 0,-1){825}}
}%
{\color[rgb]{0,0,0}\put(6001,-2536){\vector( 0,-1){900}}
}%
{\color[rgb]{0,0,0}\put(6001,-4261){\vector( 0,-1){900}}
}%
\put(6001,-511){\makebox(0,0)[b]{\smash{\SetFigFont{12}{14.4}{\rmdefault}{\mddefault}{\updefault}{\color[rgb]{0,0,0}Quotation}%
}}}
\put(6001,-2161){\makebox(0,0)[b]{\smash{\SetFigFont{12}{14.4}{\rmdefault}{\mddefault}{\updefault}{\color[rgb]{0,0,0}Sales Order}%
}}}
\put(6001,-3886){\makebox(0,0)[b]{\smash{\SetFigFont{12}{14.4}{\rmdefault}{\mddefault}{\updefault}{\color[rgb]{0,0,0}Shipping}%
}}}
\put(6001,-5611){\makebox(0,0)[b]{\smash{\SetFigFont{12}{14.4}{\rmdefault}{\mddefault}{\updefault}{\color[rgb]{0,0,0}AR Invoice}%
}}}
\end{picture}
 
\end{figure}



\subsubsection{Multiple Warehouse Example}

A customer contacts your firm and asks for a quotation for a number
of different parts. You would create a quotation and when you get
confirmation, convert it to a sales order. When you go to ship the item,
you would select the warehouse in the drop-down menu, and select the
parts to ship. One would repeat with other warehouses until the entire
order is shipped.

Then the billing department would go to the sales order and generate
the invoice. It would then be mailed to the customer.

%
\begin{figure}[hbtp]
 


\caption{Complex AR Workflow with Shipping}

\setlength{\unitlength}{3947sp}%
%
\begingroup\makeatletter\ifx\SetFigFont\undefined%
\gdef\SetFigFont#1#2#3#4#5{%
  \reset@font\fontsize{#1}{#2pt}%
  \fontfamily{#3}\fontseries{#4}\fontshape{#5}%
  \selectfont}%
\fi\endgroup%
\begin{picture}(8874,6099)(1639,-6148)
\thinlines
{\color[rgb]{0,0,0}\put(4801,-886){\framebox(2400,825){}}
}%
{\color[rgb]{0,0,0}\put(4801,-2536){\framebox(2400,825){}}
}%
{\color[rgb]{0,0,0}\put(4801,-4261){\framebox(2400,825){}}
}%
{\color[rgb]{0,0,0}\put(4801,-6061){\framebox(2400,900){}}
}%
{\color[rgb]{0,0,0}\put(6001,-886){\vector( 0,-1){825}}
}%
{\color[rgb]{0,0,0}\put(6001,-2536){\vector( 0,-1){900}}
}%
{\color[rgb]{0,0,0}\put(6001,-4261){\vector( 0,-1){900}}
}%
{\color[rgb]{0,0,0}\put(8101,-4261){\framebox(2000,825){}}
}%
{\color[rgb]{0,0,0}\put(8101,-6136){\framebox(2000,900){}}
}%
{\color[rgb]{0,0,0}\put(1651,-4261){\framebox(2400,825){}}
}%
{\color[rgb]{0,0,0}\put(5251,-2536){\line( 0,-1){375}}
\put(5251,-2911){\line(-1, 0){2250}}
\put(3001,-2911){\vector( 0,-1){525}}
}%
{\color[rgb]{0,0,0}\put(6676,-2536){\line( 0,-1){375}}
\put(6676,-2911){\line( 1, 0){2550}}
\put(9226,-2911){\vector( 0,-1){525}}
}%
{\color[rgb]{0,0,0}\put(9226,-4261){\vector( 0,-1){975}}
}%
{\color[rgb]{0,0,0}\put(3001,-4261){\line( 0,-1){1275}}
\put(3001,-5536){\vector( 1, 0){1800}}
}%
\put(6001,-511){\makebox(0,0)[b]{\smash{\SetFigFont{12}{14.4}{\rmdefault}{\mddefault}{\updefault}{\color[rgb]{0,0,0}Quotation}%
}}}
\put(6001,-2161){\makebox(0,0)[b]{\smash{\SetFigFont{12}{14.4}{\rmdefault}{\mddefault}{\updefault}{\color[rgb]{0,0,0}Sales Order}%
}}}
\put(6001,-3886){\makebox(0,0)[b]{\smash{\SetFigFont{12}{14.4}{\rmdefault}{\mddefault}{\updefault}{\color[rgb]{0,0,0}Shipping}%
}}}
\put(6001,-5611){\makebox(0,0)[b]{\smash{\SetFigFont{12}{14.4}{\rmdefault}{\mddefault}{\updefault}{\color[rgb]{0,0,0}AR Invoice}%
}}}
\put(9301,-3886){\makebox(0,0)[b]{\smash{\SetFigFont{12}{14.4}{\rmdefault}{\mddefault}{\updefault}{\color[rgb]{0,0,0}Shipping}%
}}}
\put(9301,-5761){\makebox(0,0)[b]{\smash{\SetFigFont{12}{14.4}{\rmdefault}{\mddefault}{\updefault}{\color[rgb]{0,0,0}AR Invoice}%
}}}
\put(2776,-3886){\makebox(0,0)[b]{\smash{\SetFigFont{12}{14.4}{\rmdefault}{\mddefault}{\updefault}{\color[rgb]{0,0,0}Shipping}%
}}}
\end{picture}
 
\end{figure}



\subsection{Requests for Quotation (RFQ)}

A request for quotation would be a formal document one might submit
to a vendor to ask for a quote on a product or service they might
offer. These can be generated from Purchase Orders or AP Invoices.


\subsection{Purchase Orders}

A purchase order is a confirmation that is issued to the vendor to
order the product or service. Many businesses will require a purchase
order with certain terms in order to begin work on a product. These
can be generated from RFQ's or AP Invoices.


\subsection{Receiving}

The Shipping-\textgreater Receiving screen allows you to track the
parts received from an existing purchase order. Like shipping, it
does not post an invoice but tracks the received parts in the order.


\subsection{AP Work Flow}


\subsubsection{Bookkeeper entering the received items, order completed in full}

Your company inquires about the price of a given good or service from
another firm. You submit an RFQ to the vendor, and finding that the
price is reasonable, you convert it to an order, adjust the price
to what they have quoted, and save it. When the goods are delivered
you convert the order into an AP invoice and post it.

%
\begin{figure}[hbtp]
 


\caption{Simple AP Workflow}

\setlength{\unitlength}{3947sp}%
%
\begingroup\makeatletter\ifx\SetFigFont\undefined%
\gdef\SetFigFont#1#2#3#4#5{%
  \reset@font\fontsize{#1}{#2pt}%
  \fontfamily{#3}\fontseries{#4}\fontshape{#5}%
  \selectfont}%
\fi\endgroup%
\begin{picture}(2674,4374)(5939,-4273)
\thinlines
{\color[rgb]{0,0,0}\put(7276,-736){\vector( 0,-1){1050}}
}%
{\color[rgb]{0,0,0}\put(7276,-2536){\vector( 0,-1){825}}
}%
{\color[rgb]{0,0,0}\put(5951,-736){\framebox(2650,825){}}
}%
{\color[rgb]{0,0,0}\put(5951,-2536){\framebox(2650,750){}}
}%
{\color[rgb]{0,0,0}\put(5951,-4261){\framebox(2650,900){}}
}%
\put(7201,-2161){\makebox(0,0)[b]{\smash{\SetFigFont{12}{14.4}{\rmdefault}{\mddefault}{\updefault}{\color[rgb]{0,0,0}Purchase Order}%
}}}
\put(7201,-3811){\makebox(0,0)[b]{\smash{\SetFigFont{12}{14.4}{\rmdefault}{\mddefault}{\updefault}{\color[rgb]{0,0,0}AP Invoice}%
}}}
\put(7276,-361){\makebox(0,0)[b]{\smash{\SetFigFont{12}{14.4}{\rmdefault}{\mddefault}{\updefault}{\color[rgb]{0,0,0}RFQ}%
}}}
\end{picture}
 
\end{figure}



\subsubsection{Bookkeeper entering received items, order completed in part}

Your company inquires about the price of a given good or service from
another firm, You submit an RFQ to the vendor, and finding that the
price is acceptable, you convert it into an order, adjusting the price
to what they have quoted, and save it. When some of the goods are
received, you open up the purchase order, enter the number of parts
received, convert that order into an invoice, and post it. Repeat
until all parts are received.

%
\begin{figure}[hbtp]
 


\caption{AP Workflow with Receiving}

\setlength{\unitlength}{3947sp}%
%
\begingroup\makeatletter\ifx\SetFigFont\undefined%
\gdef\SetFigFont#1#2#3#4#5{%
  \reset@font\fontsize{#1}{#2pt}%
  \fontfamily{#3}\fontseries{#4}\fontshape{#5}%
  \selectfont}%
\fi\endgroup%
\begin{picture}(2424,6024)(4789,-6073)
\thinlines
{\color[rgb]{0,0,0}\put(4801,-886){\framebox(2400,825){}}
}%
{\color[rgb]{0,0,0}\put(4801,-2536){\framebox(2400,825){}}
}%
{\color[rgb]{0,0,0}\put(4801,-4261){\framebox(2400,825){}}
}%
{\color[rgb]{0,0,0}\put(4801,-6061){\framebox(2400,900){}}
}%
{\color[rgb]{0,0,0}\put(6001,-886){\vector( 0,-1){825}}
}%
{\color[rgb]{0,0,0}\put(6001,-2536){\vector( 0,-1){900}}
}%
{\color[rgb]{0,0,0}\put(6001,-4261){\vector( 0,-1){900}}
}%
\put(6001,-2161){\makebox(0,0)[b]{\smash{\SetFigFont{12}{14.4}{\rmdefault}{\mddefault}{\updefault}{\color[rgb]{0,0,0}Purchase Order}%
}}}
\put(6001,-511){\makebox(0,0)[b]{\smash{\SetFigFont{12}{14.4}{\rmdefault}{\mddefault}{\updefault}{\color[rgb]{0,0,0}RFQ}%
}}}
\put(6001,-3886){\makebox(0,0)[b]{\smash{\SetFigFont{12}{14.4}{\rmdefault}{\mddefault}{\updefault}{\color[rgb]{0,0,0}Receiving}%
}}}
\put(6001,-5611){\makebox(0,0)[b]{\smash{\SetFigFont{12}{14.4}{\rmdefault}{\mddefault}{\updefault}{\color[rgb]{0,0,0}AP Invoice}%
}}}
\end{picture}
 
\end{figure}



\subsubsection{Receiving staff entering items}

Your company inquires about the price of a given good or service from
another firm, You submit an RFQ to the vendor, and finding that the
price is acceptable, you convert it into an order, adjusting the price
to what they have quoted, and save it. When some or all of the goods
are received, the receiving staff goes to Shipping-Receiving, locates
the purchase order, and fills in the number of items received.

The bookkeeper can then determine when all items have been received
and post the invoice at that time.

%
\begin{figure}[hbtp]
 


\caption{Complex AP Workflow}

\setlength{\unitlength}{3947sp}%
%
\begingroup\makeatletter\ifx\SetFigFont\undefined%
\gdef\SetFigFont#1#2#3#4#5{%
  \reset@font\fontsize{#1}{#2pt}%
  \fontfamily{#3}\fontseries{#4}\fontshape{#5}%
  \selectfont}%
\fi\endgroup%
\begin{picture}(8874,6099)(1639,-6148)
\thinlines
{\color[rgb]{0,0,0}\put(4801,-886){\framebox(2400,825){}}
}%
{\color[rgb]{0,0,0}\put(4801,-2536){\framebox(2400,825){}}
}%
{\color[rgb]{0,0,0}\put(4801,-4261){\framebox(2400,825){}}
}%
{\color[rgb]{0,0,0}\put(4801,-6061){\framebox(2400,900){}}
}%
{\color[rgb]{0,0,0}\put(6001,-886){\vector( 0,-1){825}}
}%
{\color[rgb]{0,0,0}\put(6001,-2536){\vector( 0,-1){900}}
}%
{\color[rgb]{0,0,0}\put(6001,-4261){\vector( 0,-1){900}}
}%
{\color[rgb]{0,0,0}\put(8101,-4261){\framebox(2000,825){}}
}%
{\color[rgb]{0,0,0}\put(8101,-6136){\framebox(2000,900){}}
}%
{\color[rgb]{0,0,0}\put(1651,-4261){\framebox(2400,825){}}
}%
{\color[rgb]{0,0,0}\put(5251,-2536){\line( 0,-1){375}}
\put(5251,-2911){\line(-1, 0){2250}}
\put(3001,-2911){\vector( 0,-1){525}}
}%
{\color[rgb]{0,0,0}\put(6676,-2536){\line( 0,-1){375}}
\put(6676,-2911){\line( 1, 0){2550}}
\put(9226,-2911){\vector( 0,-1){525}}
}%
{\color[rgb]{0,0,0}\put(9226,-4261){\vector( 0,-1){975}}
}%
{\color[rgb]{0,0,0}\put(3001,-4261){\line( 0,-1){1275}}
\put(3001,-5536){\vector( 1, 0){1800}}
}%
\put(6001,-2161){\makebox(0,0)[b]{\smash{\SetFigFont{12}{14.4}{\rmdefault}{\mddefault}{\updefault}{\color[rgb]{0,0,0}Purchase Order}%
}}}
\put(6001,-511){\makebox(0,0)[b]{\smash{\SetFigFont{12}{14.4}{\rmdefault}{\mddefault}{\updefault}{\color[rgb]{0,0,0}RFQ}%
}}}
\put(2776,-3886){\makebox(0,0)[b]{\smash{\SetFigFont{12}{14.4}{\rmdefault}{\mddefault}{\updefault}{\color[rgb]{0,0,0}Receiving}%
}}}
\put(6001,-3886){\makebox(0,0)[b]{\smash{\SetFigFont{12}{14.4}{\rmdefault}{\mddefault}{\updefault}{\color[rgb]{0,0,0}Receiving}%
}}}
\put(9301,-3886){\makebox(0,0)[b]{\smash{\SetFigFont{12}{14.4}{\rmdefault}{\mddefault}{\updefault}{\color[rgb]{0,0,0}Receiving}%
}}}
\put(6001,-5611){\makebox(0,0)[b]{\smash{\SetFigFont{12}{14.4}{\rmdefault}{\mddefault}{\updefault}{\color[rgb]{0,0,0}AP Invoice}%
}}}
\put(9301,-5761){\makebox(0,0)[b]{\smash{\SetFigFont{12}{14.4}{\rmdefault}{\mddefault}{\updefault}{\color[rgb]{0,0,0}AP Invoice}%
}}}
\end{picture}
 
\end{figure}



\subsection{Generation and Consolidation}


\subsubsection{Generation}

The Generation screen allows you to generate Purchase Orders based
on sales orders. One selects the sales orders one wants to use, and
clicks \char`\"{}Generate Purchase Orders.\char`\"{} Then one selects
clicks on the parts to order, adjusts the quantity if necessary, and
clicks \char`\"{}Select Vendor.\char`\"{} This process is repeated
for every vendor required. Then the Generate Orders button is clicked.


\subsubsection{Consolidation}

One can consolidate sales and/or purchase orders using this screen.
For the consolidation to work you must have more than one order associated
with the relevant customer or vendor.


\subsection{Reporting}

The reporting functionality in the order management is largely limited
to the ability to locate purchase orders, sales orders, RFQ's, and
quotations.


\subsection{Shipping Module: Transferring Inventory between Warehouses}

One can transfer inventory between warehouses if necessary by using
the Shipping-\textgreater Transfer Inventory screen.

\section{Fixed Assets}

One of the new features in LedgerSMB 1.3.x is fixed asset management, which
includes tracking, depreciation, and disposal.  In general, the LedgerSMB 
approach to these topics is implemented in a streamlined fashion but with
an ability to add advanced methods of depreciation later. 

\subsection{Concepts and Workflows}

Fixed asset management and accounting provides a better ability to track
the distribution of expenses relative to income.  Many fixed assets may be
depreciated so that the expense of obtaining the asset can be spread across
the usable life of that asset.  This is done in accordance with the idea of
attempting to match actual resource consumption (as expenses) with income 
in order to better gauge the financial health of the business.

\subsubsection{Fixed Assets and Capital Expenses}

Fixed assets are pieces of property that a business may own which cannot be 
easily converted into cash.  They are differentiated from liquid assets, which
include inventory, cash, bank account balances, some securities, etc.  Fixed
assets, by their nature are typically purchased and used for an extended period
of time, called the estimated usable life.  During the estimated usable life,
the fixed asset is being utilized by the business, and so we would want to 
track the expense as gradually incurred relative to the income that the asset
helps produce.  This expense is called a "capital expense" and refers to either
a purchase of a fixed asset or an expense which improves it.

Examples of capital expenses and fixed assets might include (using a pizza
place as an example):

\begin{itemize}
\item  A company vehicle
\item  Tables, chairs, ovens, etc.
\item  Changes to the leased property needed for the business.
\item  Major repairs to the company vehicle.
\end{itemize}

\subsubsection{Asset Classes}

LedgerSMB allows assets and capital expenses to be grouped together in asset
classes for easy management.  An asset class is a collection of assets which are
depreciated together, and which are depreciated in the same way.  One cannot mix
depreciation methods within a class.  Account links for the asset class are used
as default links for fixed assets, but assets may be tied to different accounts
within an asset class.

\subsubsection{Depreciation}

Depreciation is a method for matching the portion of a capital expense to
income related to it.  Expenses are depreciated so that they are spread out over
the usable life of the asset or capital expense.  Depreciation may be linear or
non-linear and maybe spread out over time or over units of production.  For
example, one might wish to depreciate a car based on miles driven, over a usable
life of, say, 100000 miles, or one might want to depreciate it based on a useful
life of five years.

LedgerSMB currently only supports variations on straight-line depreciation,
either with an estimated life measured in months or in years.

Depreciation is subject to separation of duties.  Permissions for entering and
approving depreciation reports are separate, and the GL transactions created
must currently be approved in order to show up on the books.

\subsubsection{Disposal}

Fixed assets may be disposed of through sale or abandonment.  Abandonment
generally is a method of giving the asset away.  A sale involves getting
something for the asset.

The disposal workflow is conceptually similar to the depreciation workflow
except that additionally one can track proceeds for sales, and tie in
gains/losses to appropriate income or expense accounts.

Gains are realized where the salvage value is greater than the undepreciated
value of the asset, and losses where the salvage value is less.

\subsubsection{Net Book Value}

Net book value represents the value to depreciate of fixed assets.  It is
defined as the basis value minus depreciation that has been recorded.  The basis
is further defined as the purchase value minus the estimated salvage value for
LedgerSMB purposes.  We track all capital expenses separately for depreciation
purposes, and so capital expenses which adjust value of other fixed assets have
their own net book value records.  This is separate from how capital gain and
loss might need to be tracked for tax purposes in the US.

\subsubsection{Supported Depreciation Methods}

Currently we only ship with the following variations on the straight line
depreciation method:

\begin{description}
\item[Annual Straight Line Daily]  Life is measured in years, depreciation is an
equal amount per year, divided up into equal portions daily.
\item[Annual Straight Line Monthly]  Life is measured in years, depreciation 
is an equal amount per year, divided up into equal portions each month.  This
differs from daily in that February would have less depreciation than August.
This module is more commonly used than the daily depreciation because it is
easier to calculate and thus more transparent.
\item[Whole Month Straight Line] Life is measured in months, and depreciation
occurs only per whole month.
\end{description}

\section{HR}

The HR module is currently limited to tracking employees for and their
start and end dates. It has very little other functionality. One could
build payroll systems that could integrate with it however.


\section{POS}

LedgerSMB 1.2 includes a number of components merged from Metatron Technology 
Consulting's SL-POS.  Although it is still not a perfect solution, it is greatly improved in both workflow and hardware support.  It is suitable for retail 
establishments at the moment.

\subsection{Sales Screen}

The sales screen looks very much like a normal invoice entry screen
with a few differences.

\begin{itemize}
\item The discount text field is not available, nor is the unit field.. 
\item The next part number is automatically focused when the data loads
for rapid data entry. 
\item Hot keys for the buttons are Alt-U for update, Alt-P for print, Alt-O
for post, and Alt-R for print and post. 
\item Part Groups appear at the bottom of the screen. 
\item Alt-N moves the cursor to the next free payment line.
\end{itemize}

\subsection{Possibilities for Data Entry}

\begin{itemize}
\item Barcode scanners can be used to scan items in as they are being rung
in. 
\item One could use touch screens, though this would ideally require some
custom stylesheets to make it efficient. 
\end{itemize}

\subsection{Hardware Support}

As LedgerSMB is a web-based application, the web browser usually
does not allow the page to write to arbitrary files. Therefore hardware
support for pole displays, etc. is not readily possible from the application
itself. LedgerSMB gets around this limitation by using an additional set of 
network sockets from the server to the client to control its hardware.  This 
naturally requires that other software is also running on the client.

Notes for specific types of hardware are as follows:

\begin{description}
\item [{Touch}] screens: The default stylesheet is not really usable from
a touchscreen as the items are often too small. One would need to
modify the stylesheets to ensure that the relevant items would be
reasonable. Setting down the resolution would also help. 
\item [{Receipt}] Printers: ESC/POS printers generally work in text mode.
Control sequences can be embedded in the template as necessary. 
\item [{Pole}] Displays: Generally supported.  Only the Logic Controls PD3000 is
supported out of the box, but making this work for other models ought to be
trivial.
\item [{Cash}] Drawers: These should be attached to the printer. The control
codes is then specified in the pos.conf.pl so that the command is sent to the
printer when the open till button is pushed.
\item [{Barcode}] Scanners: Most customers use decoded barcode scanners
through a keyboard wedge interface. This allows them to scan items
as if they were typing them on the keyboard. 
\end{description}

\subsection{Reports}


\subsubsection{Open Invoices}

The POS-\textgreater Open screen allows one to find any POS receipts
that are not entirely paid off.


\subsubsection{Receipts}

The POS-\textgreater Receipts screen allows one to bring up a basic
record of the POS terminals. It is not sufficient for closing the
till, however, though it may help for reconciliation.

The till column is the last component or octet of the terminal's IP
address. Therefore it is a good idea to try to avoid having IP addresses
where the last octet is the same.

All entries are grouped by date and source in this report.


\section{General Ledger}


\subsection{GL Basics}

The General Ledger is the heart of LedgerSMB. Indeed, LedgerSMB
is designed to be as close as possible to a software equivalent of
a paper-based accounting program (but with no difference between the
General Ledger and General Journal).


\subsubsection{Paper-based accounting systems and the GL}

In order to understand the principle of the General Ledger, one must
have a basic understanding of the general process of bookkeeping using
double-entry paper-based accounting systems.

Normally when a transaction would be recorded, it would first be recorded
in the ``General Journal'' which would contain detailed
information about the transaction, notes, etc. Then the entries from
the General Journal would be transcribed to the General Ledger, where
one could keep closer tabs on what was going on in each account.

In the general journal, all transactions are listed chronologically
with whatever commentary is deemed necessary, while in the general
ledger each account has its own page and transactions are recorded
in a simple and terse manner. The General Journal is the first place
the transaction is recorded and the General Ledger is the last.

At the end of the accounting period, the GL transactions would be
summarized into a trial balance and this would be used for creating
financial statements and closing the books at the end of the year.


\subsubsection{Double Entry Examples on Paper}

Let us say that John starts his business with an initial investment
of \$10,000.

This is recorded in the General Journal as follows (in this example,
suppose it is page 1):

\begin{tabular}{|l|l|l|r|r|}
\hline 
Date &
Accounts and Explanation &
Ref &
DEBIT &
CREDIT \tabularnewline
\hline 
March 1 &
Checking Account &
1060 &
10000.00 &
\tabularnewline
&
John Doe Capital &
3011 &
&
10000.00\tabularnewline
&
John Doe began a business &
&
&
\tabularnewline
&
with an investment of &
&
&
\tabularnewline
&
\$10000 &
&
&
\tabularnewline
\hline
\end{tabular}\medskip{}


This would then be transcribed into two pages of the General Ledger.
The first page might be the Checking Account page:\medskip{}


\begin{tabular}{|l|l|l|r|l|l|l|r|}
\hline 
DATE &
EXPLANATION &
REF. &
DEBITS &
DATE &
EXPLANATION &
REF. &
CREDITS\tabularnewline
\hline 
March 1 &
&
J1 &
10000.00 &
&
&
&
\tabularnewline
\hline
\end{tabular}\medskip{}


On the John Doe Capital page, we would add a similar entry:\medskip{}


\begin{tabular}{|l|l|l|r|l|l|l|r|}
\hline 
DATE &
EXPLANATION &
REF. &
DEBITS &
DATE &
EXPLANATION &
REF. &
CREDITS\tabularnewline
\hline 
&
&
&
&
March 1 &
&
J1 &
10000.00\tabularnewline
\hline
\end{tabular}\medskip{}



\subsubsection{The GL in LedgerSMB}

The paper-based accounting procedure works well when one is stuck
with paper recording requirements but it has one serious deficiency---
all of this transcribing creates an opportunity for errors.

Relational databases relieve the need for such transcription as it
is possible to store everything physically in a way similar to the
way a General Journal is used in the paper-based systems and then
present the same information in ways which are more closely related
to the General Ledger book.

This is the exact way that the General Ledger is used in LedgerSMB.
The actual data is entered and stored as if it was a general journal,
and then the data can be presented in any number of different ways.

All modules of LedgerSMB that involve COA accounts store their data
in the General Ledger (it is a little more complex than this but this
is very close to the actual mechanism).


\subsection{Cash Transfer}

The simplest form of GL entry in LedgerSMB is the Cash-\textgreater
Transfer screen. This screen shows two transaction lines, and fields
for reference, department, description, and notes.

The field descriptions are as follows:

\begin{description}
\item [{Reference}] refers to the source document for the transfer. One
can use transfer sheets, bank receipt numbers, etc for this field. 
\item [{Description}] is optional but really should be filled in. It ought
to be a description of the transaction. 
\item [{Notes}] provide supplemental information for the transaction. 
\item [{FX}] indicates whether foreign exchange is a factor in this transaction. 
\item [{Debit}] indicates money going \textbf{into} the asset account. 
\item [{Credit}] indicates money coming \textbf{out} of the asset account. 
\item [{Source}] is the source document for that portion of the transaction. 
\item [{Memo}] lists additional information as necessary.
\item [{Project}] allows you to assign this line to a project. 
\end{description}
The credit and debit options seem to be the opposite of what one would
think of concerning one's bank account. The reason is that your bank
statement is done from the bank's point of view.  Your bank account balance
is an asset to you and therefor you show it as having a debit balance, but
to the bank it is money they owe you and so they show it as having a credit
balance.

Note that in this screen, when an item is updated, it will reduce
the number of lines to those already filled in plus an extra line
for the new line in the data entry.


\subsection{GL Transactions}

The GL Transaction screen (General Ledger-\textgreater Add Transaction)
is identical to the Cash Transfer screen with the exception that it
starts with nine lines instead of two. Otherwise, they are identical.

Again, one must be careful with debits and credits. Often it is easy
to get confused. It is generally worth while to go back to the principle
that one tracks them with regard to their impact on the equity accounts.
So expenses are credits because they debit the equity accounts, and
income is a debit because it credits the retained earning equity account.


\subsection{Payroll as a GL transaction}

Currently payroll must be done as a GL transaction. The attempts to
create a payroll system that would ship with LSMB have largely stalled.

Most customers running their businesses will have an idea of how to
do this.

%
\begin{figure}[hbtp]
 


\caption{Payroll as a GL Transaction (Purely fictitious numbers)}

\begin{tabular}{|l|r|r|}
\hline 
Account &
Debit &
Credit \tabularnewline
5101 Wages and Salaries &
500 &
\tabularnewline
2032 Accrued Wages &
&
450 \tabularnewline
2033 Fed. Income Tax wthd &
&
30 \tabularnewline
2034 State Inc. Tax. wthd &
&
15 \tabularnewline
2035 Social Security wthd &
&
3 \tabularnewline
2036 Medicare wthd &
&
2 \tabularnewline
2032 Accrued Wages &
450 &
\tabularnewline
1060 Checking Acct &
&
450 \tabularnewline
\hline
\end{tabular}
\end{figure}



\subsection{Reconciliation}

To reconcile an account (say, when one would get a checking account
statement), one would go to cash/reconciliation, and check off the
items that have cleared. One can then attempt to determine where any
errors lie by comparing the total on the statement with the total
that LSMB generates.

This can be done for other accounts too, such as petty cash.%
\footnote{Petty cash denotes a drawer of cash that is used to pay small
expenses. When an expense is paid, it is recorded on a slip of paper that is
stored for reconciliation purposes.}  Some users even reconcile liability
accounts and the likes.

In LedgerSMB 1.3.x, the reconciliation framework has been completely rewritten
to allow for easier reconciliation especially where there are large numbers of
transactions.  It is now possible to reconcile accounts with thousands of
transactions per month, and to track what was reconciled in each report.
Reconciliation is now also subject to separation of duties, allowing a
reconciliation report to be committed to the books only when approved.

The reconciliation screen is now divided into four basic parts.  At the top is a
header with information about which account is being reconcilied, ending
balances, variances, etc.  Then comes a list of cleared transactions, as well
as totals of debits and credits.

After this comes a (usually empty) list of failed matches when the file import
functionality is used.  These again list the source, then the debits/credits in
LedgerSMB and the debits/credits in the bank import.

Finally there is a list of uncleared transactions.  To move error a transaction
from the error or uncleared section into the cleared section, check one or more
off and click update.  Update also checks for new uncleared transactions in the
reconciliation period, and it saves the current report so it can be continued
later.

\subsubsection{File Import Feature}

1.3.x has a plugin model that allows one to write parsers against a variety of
file formats, one or more per account.  The file can be placed in the
LedgerSMB/Reconciliation/CSV/Formats directory, and the function must be called
parse\_ followed by the account id.  The function must be in the
LedgerSMB::Reconciliation::CSV namespace. 

This obviously has a few limitations.  Hosting providers might want to start the
account tables 10000 apart as far as the initial id's go or provide dedicated
LedgerSMB instances per client. 

\subsection{Reports}

The most flexible report in LedgerSMB is the GL report because it
has access to the entire set of financial transactions of a business.
Every invoice posted, payment made or received, etc. can be located
here.

The search criteria include:

\begin{description}
\item [{Reference}] is the invoice number, or other reference number associated
with the transaction. 
\item [{Source}] is the field related to the source document number in
a payment or other transaction.%
\footnote{Source documents are things like receipts, canceled checks, etc. that
can be used to verify the existence and nature of a transaction.%
} 
\item [{Memo}] relates to the memo field on a payment.
\item [{Department}] can be used to filter results by department. 
\item [{Account}] Type can be used to filter results by type of account
(Asset, Liability, etc.) 
\item [{Description}] can be used to filter by GL description or by
customer/vendor name. 
\end{description}
The actual format of the report looks more like what one would expect
in a paper accounting system's general journal than a general ledger
per se. A presentation of the data that is more like the paper general
ledger is found in the Chart of Accounts report.


\subsubsection{GL as access to almost everything else}

The GL reports can be used to do all manner of things. One can determine,
for example, which AP invoice or transaction was paid with a certain
check number or which invoice by a specific customer was paid by a specific
check number.


\section{Recurring Transactions}

Any transaction or invoice may be repeated a number of times in regular
intervals. To schedule any GL, AR, or AP transaction or invoice, click
the schedule button.

In general the reference number should be left blank as this will
force LedgerSMB to create a new invoice or transaction number for
each iteration. The rest of the options are self-explanatory. Note
that a blank number if iterations will result in no recurrences of
the transaction.

To process the recurring transactions, click on the Recurring Transactions
option on the main menu select the ones you want to process and click
\char`\"{}Process Transactions.\char`\"{}

An enhanced recurring transaction interface is forthcoming from the LedgerSMB
project.

\section{Financial Statements and Reports}

Financial statements and reports are a very important part of any
accounting system. Accountants and business people rely on these reports
to determine the financial soundness of the business and its prospects
for the next accounting period.


\subsection{Cash v. Accrual Basis}

Financial statements, such as the Income Statement and Balance Sheet
can be prepared either on a cash or accrual basis. In cash-basis accounting,
the income is deemed earned when the customer pays it, and the expenses
are deemed incurred when the business pays them.

There are a number of problems with cash-basis accounting from a business
point of view. The most serious is that one can misrepresent the wellbeing
of a business by paying a large expense after a deadline. Thus cash-basis
accounting does not allow one to accurately pair the income with the
related expense as these are recorded at different times. If one cannot
accurately pair the income with the related expense, then financial
statements cannot be guaranteed to tell one much of anything about
the well-being of the business.

In accrual basis accounting, income is considered earned when the
invoice is posted, and expenses are considered incurred at the time
when the goods or services are delivered to the business. This way,
one can pair the income made from the sale of a product with the expense
incurred in bringing that product to sale. This pairing allows for
greater confidence in business reporting.


\subsection{Viewing the Chart of Accounts and Transactions}

The Reports--\textgreater Chart of Accounts will provide the chart
of accounts along with current totals in each account.

If you click on an account number, you will get a screen that allows
you to filter out transactions in that account by various criteria.
One can also include AR/AP, and Subtotal in the report.

The report format is similar to that of a paper-based general ledger.


\subsection{Trial Balance}


\subsubsection{The Paper-based function of a Trial Balance}

In paper-based accounting systems, the accountant at the end of the
year would total up the debits and credits in every account and transfer
them onto another sheet called the trial balance. The accountant would
check to determine that the total debits and credits were equal and
would then transfer this information onto the financial statements.
It was called a trial balance because it was the main step at which
the error-detection capabilities of double-entry accounting systems
were used.


\subsubsection{Running the Trial Balance Report}

This report is located under Reports --\textgreater Trial Balance.
One can filter out items by date, accounting period, or department.
One can run the report by accounts or using GIFI classifications to
group accounts together.

From this report, you can click on the account number and see all
transactions on the trial balance as well as whether or not they have
been reconciled.


\subsubsection{What if the Trial Balance doesn't Balance?}

If the trial balance does not balance, get technical support immediately.
This usually means that transactions were not entered properly. Some
may have been out of balance, or some may have gone into non-existent
accounts (believe it or not, LedgerSMB does not check this latter
issue).


\subsubsection{Trial Balance as a Summary of Account Activity}

The trial balance offers a glance at the total activity in every account.
It can provide a useful look at financial activity at a glance for
the entire business.


\subsubsection{Trial Balance as a Budget Planning Tool}

By filtering out departments, one can determine what a department
earned and spent during a given financial interval. This can be used
in preparing budgets for the next accounting period.


\subsection{Income Statement}

The Income Statement is another tool that can be used to assist with
budgetary planning as well as provide information on the financial
health of a business.

The report is run from Reports--\textgreater Income Statement. The
report preparation screen shows the following fields:

\begin{description}
\item [{Department}] allows you to run reports for individual departments.
This is useful for budgetary purposes. 
\item [{Project}] allows you to run reports on individual projects. This
can show how profitable a given project was during a given time period. 
\item [{From}] and To allow you to select arbitrary from and to dates. 
\item [{Period}] allows you to specify a standard accounting period. 
\item [{Compare to}] fields allow you to run a second report for comparison
purposes for a separate range of dates or accounting period. 
\item [{Decimalplaces}] allows you to display numbers to a given precision. 
\item [{Method}] allows you to select between accrual and cash basis reports. 
\item [{Include}] in Report provides various options for reporting. 
\item [{Accounts}] allows you to run GIFI reports instead of the standard
ones. 
\end{description}
The report shows all income and expense accounts with activity during
the period when the report is run, the balances accrued during the
period, as well as the total income and expense at the bottom of each
section. The total expense is subtracted from the total income to
provide the net income during the period. If there is a loss, it appears
in parentheses.


\subsubsection{Uses of an Income Statement}

The income statement provides a basic snapshot of the overall ability
of the business to make money. It is one of the basic accounting statements
and is required, for example, on many SEC forms for publicly traded
firms.

Additionally, businessmen use the income statement to look at overall
trends in the ability of the business to make money. One can compare
a given month, quarter, or year with a year prior to look for trends
so that one can make adjustments in order to maximize profit.

Finally, these reports can be used to provide a look at each department's
performance and their ability to work within their budget. One can
compare a department or project's performance to a year prior and
look for patterns that can indicate problems or opportunities that
need to be addressed.


\subsection{Balance Sheet}

The balance sheet is the second major accounting statement supported
by LedgerSMB. The balance sheet provides a snapshot of the current
financial health of the business by comparing assets, liabilities,
and equity.

In essence the balance sheet is a statement of the current state of
owner equity. Traditionally, it does not track changes in owner equity
in the same way the Statement of Owner Equity does.

The Balance Sheet report preparation screen is much simpler than the
Income Statement screen. Balance sheets don't apply to projects, but
they do apply to departments. Also, unlike an income statement, a
balance sheet is fixed for a specific date in time. Therefore one
does not need to select a period.

The fields in creating a balance sheet are:

\begin{description}
\item [{Department}] allows you to run separate balance sheets for each
department. 
\item [{As}] at specifies the date. If blank this will be the current date. 
\item [{Compare to}] specifies the date to compare the balance sheet to. 
\item [{Decimalplaces}] specifies the number of decimal places to use. 
\item [{Method}] selects between cash and accrual basis. 
\item [{Include}] in report allows you to select supplemental information
on the report. 
\item [{Accounts}] allows you to select between standard and GIFI reports. 
\end{description}
The balance sheet lists all asset, liability, and equity accounts
with a balance. Each category has a total listed, and the total of
the equity and liability accounts is also listed.

The total assets should be equal to the sum of the totals of the liability
and equity accounts.


\subsection{What if the Balance Sheet doesn't balance?}

Get technical support immediately, This may indicate that out of balance
transactions were entered or that transactions did not post properly.


\subsection{No Statement of Owner Equity?}

The Statement of Owner Equity is the one accounting statement that
LedgerSMB does not support. However, it can be simulated by running
a balance sheet at the end of the time frame in question and comparing
it to the beginning. One can check this against an income statement
for the period in question to verify its accuracy. The statement of
owner equity is not as commonly used now as it once was.


\section{The Template System}

LedgerSMB allows most documents to be generated according to a template
system. This allows financial statements, invoices, orders, and the
like to be customized to meet the needs of most businesses. Company
logos can be inserted, the format can be radically altered, one can
print letters to be included with checks to vendors instead of the
checks themselves, and the like. In the end, there is very little
that cannot be accomplished regarding modification of these documents
with the template system.

One can define different templates for different languages, so that
a customer in Spain gets a different invoice than a customer in Canada.

LedgerSMB provides templates in a variety of formats including text, html,
LaTeX, Excel, and Open Document Spreadsheat.  Each of these is processed using
TemplateToolkit\footnote{See \url{http://template-toolkit.org/} for more
information and documentation.} with a few specific modifications:

\begin{itemize}
\item start tag is \textless?lsmb and end tag is ?\textgreater
\item text(string) is available as a function in the templates for localization
purposes.
\item gettext(string, language) can be used to translate a string into a
specified language rather than the default language (useful for multi-lingual
templates).
\end{itemize}

Additionally the UI/lib/ui-header.html can be used to provide standardized
initial segments for HTML documents and inputs can be entered via interfaces in
the UI/lib/elements.html template.

\subsubsection{What is \LaTeX{}\ ?}

\LaTeX{}\ (pronounced LAY-tech) is an extension on the \TeX{}\ typesetting
system. It largely consists of a set of macros that allow one to focus
on the structure of the document while letting the \TeX{}\ engine
do the heavy lifting in terms of determining the optimal formatting
for the page. \LaTeX{}\ is used in a large number of academic journals
(including those of the American Mathematics Association). It is available
at \url{http://www.tug.org} and is included in most Linux distributions.

Like HTML, \LaTeX{}\ uses plain text documents to store the formatting
information and then when the document is rendered, attempts to fit
it onto a page. \LaTeX{}\
supports the concept of stylesheets, allowing one to separate content
from format, and this feature is used in many higher-end applications,
like journal publication.

Unlike HTML, \LaTeX{}\ is a complete though simple programming language
that allows one to redefine internals of the system for formatting
purposes.

This document is written in \LaTeX{}.


\subsubsection{Using \LyX{} to Edit \LaTeX{}\ Templates}

\LyX{} is a synchronous \LaTeX{}\ editor that runs on Windows, UNIX/Linux,
and Mac OS X. It requires an installed \LaTeX{}-2e implementation
and can be obtained at \url{http://www.lyx.org}. Like the most common
\LaTeX{}\ implementations, it is open source and is included with most 
Linux distributions.

\subsection{Customizing Logos}

\LaTeX{}\ requires different formats of logos depending on whether
the document is going to be generated as a PDF or as postscript. Postscript
requires an embedded postscript graphic, while PDF requires any type
of graphic other than embedded postscript. Usually one uses a PNG's
for PDF's, though GIF's could be used as well. The logo for a \LaTeX{}\ document
must be fully qualified as to its path.

HTML documents can have logos in many different formats. PNG's are
generally preferred for printing reasons. The image can be stored
anywhere and merely referenced in the HTML.

Note: Always test the an invoice with images to ensure that
the rest of the page format is not thrown off by it.


\subsection{How are They Stored in the Filesystem?}

The template directory (``templates'' in the root
LedgerSMB install directory) contains all the root templates used
by LedgerSMB.  The default templates are stored in the demo folder.

Inside the templates directory are one or more subdirectories where the 
relevant
templates have been copied as default language templates for the user.
Many users can use the same user directory (which bears the name of
the LedgerSMB username). Within this directory are more subdirectories
for translated templates, one for each language created.  If the requested
language template is not found, LedgerSMB will attempt to translate the one in
the main folder.


\subsection{Upgrade Issues}

When LedgerSMB is upgraded, the templates are not replaced. This
is designed to prevent the upgrade script from overwriting changes
made during the course of customizing the templates.

Occasionally, however, the data model changes in a way which can cause
the templates to stop printing certain information. When information
that was showing up before an upgrade stops showing up, one can either
upgrade the templates by copying the source template over the existing
one, or one can edit the template to make the change.

\section{An Introduction to the CLI for Old Code}

This section applies to the following Perl scripts:

\begin{description}
\item[am.pl]  System Management
\item[ap.pl]  Accounts Payable Transactions and Reports 
\item[ar.pl]  Accounts Receivable Transactions and Reports
\item[bp.pl]  Batch Printing
\item[ca.pl]  Chart of Accounts (some functions, others are in the accounts.pl
script)
\item[gl.pl]  General Ledger Reports and Journal Entry
\item[ic.pl]  Inventory control
\item[ir.pl]  Invoices Received (vendor invoices)
\item[is.pl]  Invoices Shipped (sales invoices)
\item[jc.pl]  Job costing (timecards)
\item[oe.pl]  Order Entry
\item[pe.pl]  Projects
\item[ps.pl]  Point of Sale
\item[rp.pl]  Reports
\end{description}

\subsection{Conventions}

The command-line API will be referred to as the API.

\subsection{Preliminaries}

All scripts included in the documentation can also be found in the doc/samples
directory.

Consider a simple example:

 cd /usr/local/ledger-smb
./ct.pl "login=name\&path=bin\&password=xxxxx\&action=search\&db=customer"

The cd command moves your terminal session's current working directory into
the main LedgerSMB directory. Then the LedgerSMB perl script ct.pl is called 
with one long line as an argument. The argument is really several variable=value pairs
separated by ampersands (\&). The value for the login variable is the username
that LedgerSMB is to use, and the value for the password variable is the plaintext password.

To build our examples we will use a username of "clarkkent" who has a password
of "lOis,lAn3".

 cd /usr/local/ledger-smb
./ct.pl "login=clarkkent\&path=bin\&password=lOis,lAn3\&action=search\&db=customer"

If we execute these commands we will get the html for the search form for 
the customer database. This result isn't useful in itself, but it shows we 
are on the right track.


\subsection{First Script: lsmb01-cli-example.sh}

With a working example, we can start to build reproducible routines that we can grow
to do some useful work.

This is a bash script which:

\begin{enumerate}
\item sets NOW to the current working directory
\item prompts for and reads your LedgerSMB login
\item prompts for and reads (non-echoing) your LedgerSMB password
\item changes directory to /usr/local/ledger-smb
\item constructs login and logout commands and a transaction command
\item logins into ledger-smb (in a real program, output would be checked for
    success or failure)
\item executes the transaction
\item logs out of ledger-smb (although this is not necessary)
\item returns to the original working directory
\item exits
\end{enumerate}

Running lsmb01-cli-example.sh produces:

\$ lsmb01-cli-example.sh 

LedgerSMB login: clarkkent

LedgerSMB password: 

\begin{verbatim}
<body>

<form method=post action=ct.pl>

<input type=hidden name=db value=customer>

<table width=100%>
  <tr>
    <th class=listtop>Search</th>
.
.
.
\end{verbatim}

A script like this would work well for simple batch transactions, but
bash is not a very friendly language for application programming.


A nicer solution would be to use a language such as perl to drive the
command line API.

\subsubsection{Script 1 (Bash)}
\begin{verbatim}
#!/bin/bash
#######################################################################
#
# lsmb01-cli-example.sh
# Copyright (C) 2006. Louis B. Moore
#
# $Id: $
#
# This program is free software; you can redistribute it and/or modify
# it under the terms of the GNU General Public License as published by
# the Free Software Foundation; either version 2 of the License, or
# (at your option) any later version.
#
# This program is distributed in the hope that it will be useful,
# but WITHOUT ANY WARRANTY; without even the implied warranty of
# MERCHANTABILITY or FITNESS FOR A PARTICULAR PURPOSE.  See the
# GNU General Public License for more details.
# You should have received a copy of the GNU General Public License
# along with this program; if not, write to the Free Software
# Foundation, Inc., 675 Mass Ave, Cambridge, MA 02139, USA.
#
#######################################################################

NOW=`pwd`

echo -n "Ledger-SMB login: "
read LSLOGIN
echo

echo -n "Ledger-SMB password: "
stty -echo
read LSPWD
stty echo
echo

ARG="login=${LSLOGIN}&password=${LSPWD}&path=bin&action=search&db=customer"

LGIN="login=${LSLOGIN}&password=${LSPWD}&path=bin&action=login"
LGOT="login=${LSLOGIN}&password=${LSPWD}&path=bin&action=logout"

cd /usr/local/ledger-smb

./login.pl $LGIN 2>&1  > /dev/null
./ct.pl $ARG
./login.pl $LGOT 2>&1  > /dev/null

cd $NOW

exit 0
\end{verbatim}


\subsection{Second Script: lsmb02-cli-example.pl}

Our second script is written in perl and logs you in but it still uses the API 
in its simplest form, that is, it builds commands and then executes them. This 
type of script can be used for more complex solutions than the simple bash script 
above, though it is still fairly limited. If your needs require, rather than have 
the script build and then execute the commands it could be written to generate a 
shell script which is executed by hand. 

This script begins by prompting for your LedgerSMB login and password. Using
the supplied values a login command is constructed and passed into the runLScmd
subroutine. runLScmd changes directory to /usr/local/ledger-smb/ for the length 
of the subroutine. It formats the command and executes it and returns both the 
output and error information to the caller in a scalar.

The script checks to see if there was an error in the login, exiting if there was.

Next, the script reads some records which are stored in the program following the
\_\_END\_\_ token. It takes each record in turn, formats it then feeds each transaction
through runLScmd and looks at the results for a string that signals success.

Once all the transactions are processed, runLScmd is called one last time to
logout and the script exits.

\subsubsection{Script 2 (Perl)}
\begin{verbatim}
#!/usr/bin/perl -w
#
#  File:         lsmb02-cli-example.pl
#  Environment:  Ledger-SMB 1.2.0+
#  Author:       Louis B. Moore
#
#  Copyright (C)   2006  Louis B. Moore
#
#  This program is free software; you can redistribute it and/or
#  modify it under the terms of the GNU General Public License
#  as published by the Free Software Foundation; either version 2
#  of the License, or (at your option) any later version.
#
#  This program is distributed in the hope that it will be useful,
#  but WITHOUT ANY WARRANTY; without even the implied warranty of
#  MERCHANTABILITY or FITNESS FOR A PARTICULAR PURPOSE.  See the
#  GNU General Public License for more details.
#
#  You should have received a copy of the GNU General Public License
#  along with this program; if not, write to the Free Software
#  Foundation, Inc., 59 Temple Place - Suite 330, Boston, MA  02111-1307, USA.
#
#  Revision:
#       $Id$
#
#

use File::chdir;
use HTML::Entities;


print "\n\nLedger-SMB login: ";
my $login = <STDIN>;
chomp($login);


print "\nLedger-SMB password: ";
system("stty -echo");
my $pwd = <STDIN>;
system("stty echo");
chomp($pwd);
print "\n\n";

$cmd = "login=" . $login . '&password=' . $pwd . '&path=bin&action=login';

$signin = runLScmd("./login.pl",$cmd);

if ( $signin =~ m/Error:/ ) {

	print "\nLogin error\n";
	exit;

}


while (<main::DATA>) {

	chomp;
	@rec = split(/\|/);

	$arg = 'path=bin/mozilla&login=' . $login . '&password=' . $pwd .
		'&action='       . escape(substr($rec[0],0,35)) .
		'&db='           . $rec[1] .
		'&name='         . escape(substr($rec[2],0,35)) .
		'&vendornumber=' . $rec[3] .
		'&address1='     . escape(substr($rec[4],0,35)) .
		'&address2='     . escape(substr($rec[5],0,35)) .
		'&city='         . escape(substr($rec[6],0,35)) .
		'&state='        . escape(substr($rec[7],0,35)) .
		'&zipcode='      . escape(substr($rec[8],0,35)) .
		'&country='      . escape(substr($rec[9],0,35)) .
		'&phone='        . escape(substr($rec[10],0,20)) .
		'&tax_2150=1' .
		'&taxaccounts=2150' .
		'&taxincluded=0' .
		'&terms=0';

	$rc=runLScmd("./ct.pl",$arg);

	if ($rc =~ m/Vendor saved!/) {

		print "$rec[2] SAVED\n";

	} else {

		print "$rec[2] ERROR\n";

	}

}


$cmd = "login=" . $login . '&password=' . $pwd . '&path=bin&action=logout';

$signin = runLScmd("./login.pl",$cmd);

if ( $signin =~ m/Error:/ ) {

    print "\nLogout error\n";

}

exit;


#*******************************************************
# Subroutines
#*******************************************************


sub runLScmd {

    my $cmd  = shift;
    my $args = shift;
    my $i    = 0;
    my $results;

    local $CWD = "/usr/local/ledger-smb/";

    $cmd = $cmd . " \"" . $args . "\"";

    $results = `$cmd 2>&1`;

    return $results;

}

sub escape {

    my $str = shift;

    if ($str) {

	decode_entities($str);
	$str =~ s/([^a-zA-Z0-9_.-])/sprintf("%%%02x", ord($1))/ge;
    }

    return $str;

}


#*******************************************************
# Record Format
#*******************************************************
#
# action | db | name | vendornumber | address1 | address2 | city | state | zipcode | country | phone
#

__END__
save|vendor|Parts are Us|1377|238 Riverview|Suite 11|Cheese Head|WI|56743|USA|555-123-3322|
save|vendor|Widget Heaven|1378|41 S. Riparian Way||Show Me|MO|39793|USA|555-231-3309|
save|vendor|Consolidated Spackle|1379|1010 Binary Lane|Dept 1101|Beverly Hills|CA|90210|USA|555-330-7639 x772|

 	  	 
\end{verbatim}


\clearpage 


\part{Technical Overview}


\section{Basic Architecture}

LedgerSMB is a web-based Perl program that interfaces with PostgreSQL
using the relevant Perl modules. The code is well partitioned, and
the main operation modules are written in an object oriented way.


\subsection{The Software Stack}

%
\begin{figure}[hbtp]
 \label{fig-sl-stack} \setlength{\unitlength}{3947sp}%
%
\begingroup\makeatletter\ifx\SetFigFont\undefined%
\gdef\SetFigFont#1#2#3#4#5{%
  \reset@font\fontsize{#1}{#2pt}%
  \fontfamily{#3}\fontseries{#4}\fontshape{#5}%
  \selectfont}%
\fi\endgroup%
\begin{picture}(4824,4824)(3589,-5173)
\thinlines
{\color[rgb]{0,0,0}\put(3601,-1261){\framebox(4800,900){}}
}%
{\color[rgb]{0,0,0}\put(6001,-3961){\framebox(2400,1200){}}
}%
{\color[rgb]{0,0,0}\put(3601,-5161){\framebox(4800,1200){}}
}%
{\color[rgb]{0,0,0}\put(3601,-2761){\framebox(4800,1500){}}
}%
{\color[rgb]{0,0,0}\put(3601,-3961){\framebox(2400,1200){}}
}%
\put(7201,-3361){\makebox(0,0)[b]{\smash{\SetFigFont{17}{20.4}{\rmdefault}{\mddefault}{\updefault}{\color[rgb]{0,0,0}PostgreSQL}%
}}}
\put(6001,-4561){\makebox(0,0)[b]{\smash{\SetFigFont{17}{20.4}{\rmdefault}{\mddefault}{\updefault}{\color[rgb]{0,0,0}Operating System}%
}}}
\put(6001,-811){\makebox(0,0)[b]{\smash{\SetFigFont{17}{20.4}{\rmdefault}{\mddefault}{\updefault}{\color[rgb]{0,0,0}LedgerSMB}%
}}}
\put(4201,-3361){\makebox(0,0)[lb]{\smash{\SetFigFont{17}{20.4}{\rmdefault}{\mddefault}{\updefault}{\color[rgb]{0,0,0}Apache}%
}}}
\put(6001,-1936){\makebox(0,0)[b]{\smash{\SetFigFont{17}{20.4}{\rmdefault}{\mddefault}{\updefault}{\color[rgb]{0,0,0}Perl}%
}}}
\end{picture}
 


\caption{The LedgerSMB software stack in a Typical Implementation}
\end{figure}


LedgerSMB runs in a Perl interpreter. I do not currently know if
it is possible to run it with Perl2C or other language converters
to run in other environments. However, except for high-capacity environments,
Perl is a good language choice for this sort of program.

LedgerSMB used to support DB2 and Oracle as well as PostgreSQL. However,
currently some of the functionality is implemented using PostgreSQL
user-defined functions. These would need to be ported to other database
managers in order to make the software work on these. It should not
be too hard, but the fact that it has not been done yet may mean that
there is no real demand for running the software under other RDBMS's.

One can substitute other web servers for Apache. Normally LedgerSMB
is run as a CGI program but it may be possible to run it in the web
server process (note that this may not be entirely thread-safe).

The operating system can be any that supports a web server and Perl
(since PostgreSQL need not run on the same system). However, there
are a few issues running LedgerSMB on Windows (most notably in trying
to get Postscript documents to print properly).

On the client side, any web-browser will work. Currently, the layout
is different for Lynx (which doesn't support frames), and the layout
is not really useful under eLinks (the replacement for Lynx which
does support frames). Some functionality requires Javascript to work
properly, though the application is usable without these features.


\subsection{Capacity Planning}

Some companies may ask how scalable LedgerSMB is. In general, it
is assumed that few companies are going to have a need for a high-concurrency
accounting system. However, with all the features available in LedgerSMB,
the staff that may have access to some of the application may be senior
enough to make the question worthwhile.

As of 1.3, there are users with databases containing over a million financial
transactions, thousands of customers and vendors, and a full bookkeeping
department.  In general, 1.3 is considered scalable for midsize businesses.

In general if you are going to have a large number of users for the software, or
large databases, we'd generally suggest running the database on a separate
server, since this makes it easier to isolate and address performance issues.
Additionally, the database server is the hardest part to scale horizontally and
so you want to put more hardware there than with the database server.

\subsubsection{Scalability Strategies}

LedgerSMB is a fairly standard web-based application. However,
sometimes the database schema changes during upgrades. In these cases,
it becomes impossible to use different versions of the software against
the same database version safely. LedgerSMB checks the version of
the database and if the version is higher than the version of the
software that is running, will refuse to run.

Therefore although one strategy might be to run several front-end
web servers with LedgerSMB, in reality this can be a bit of a problem.
One solution is to take half of the front-end servers off-line while
doing the initial upgrade, and then take the other offline to upgrade
when these are brought back online.

The database manager is less scalable in the sense that one cannot
just add more database servers and expect to carry on as normal. However,
aside from the known issues listed below, there are few performance
issues with it. If complex reports are necessary, these can be moved
to a replica database (perhaps using Slony-I or the streaming repliaction of
PostgreSQL 9.x).

If this solution is insufficient for database scalability, one might
be able to move staff who do not need real-time access to new entries
onto a PG-Pool/Slony-I cluster where new transactions are entered
on the master and other data is looked up on the replica. In certain
circumstances, one can also offload a number of other queries from
the master database in order to minimize the load. LedgerSMB has
very few issues in the scalability of the application, and those we find, we
correct as quickly as possible.


\subsubsection{Database Maintenance}

PostgreSQL uses a technique called Multi-version Concurrency Control
(MVCC) to provide a snapshot of the database at the beginning of a
statement or transaction (depending on the transaction isolation level).
When a row is updated, PostgreSQL leaves the old row in the database,
and inserts a new version of that row into the table. Over time, unless
those old rows are removed, performance can degrade as PostgreSQL
has to search through all the old versions of the row in order to
determine which one ought to be the current one.

Due to the way the SQL statements are executed in LedgerSMB, many
inserts will also create a dead row.

A second problem occurs in that each transaction is given a transaction
id. These id's are numbered using 32-bit integers. If the transaction
id wraps around (prior to 8.1), data from transactions that appear
(due to the wraparound) to be in the future suddenly becomes inaccessible.
This problem was corrected in PostgreSQL 8.1, where the database will
refuse to accept new transactions if the transaction ID gets too close
to a wraparound. So while the problem is not as serious in 8.1, the
application merely becomes inaccessible rather than displaying apparent
data loss. Wraparound would occur after about a billion transactions
between all databases running on that instance of PostgreSQL.

Prior to 8.1, the main way to prevent both these problems was to run
a periodic vacuumdb command from cron (UNIX/Linux) or the task scheduler
(Windows). In 8.1 or later, autovacuum capabilities are part of the
back-end and can be configured with the database manager. See the
PostgreSQL documentation for treatment of these subjects.

In general, if performance appears to be slowly degrading, one should
try to run vacuumdb -z from the shell in order to attempt to reclaim
space and provide the planner with accurate information about the
size and composition of the tables. If this fails, then one can go
to other methods of determining the bottleneck and what to do about
it.


\subsubsection{Known issues}

There are no known issues in PostgreSQL performance as relates to LedgerSMB
performance in supported versions.

\section{Customization Possibilities}

LedgerSMB is designed to be customized relatively easily and rapidly.
In general, the source code is well written and compartmentalized.
This section covers the basic possibilities involving customization.


\subsection{Brief Guide to the Source Code}

LedgerSMB is an application with over 120000 lines of code.\footnote{The line
count provided by Ohloh.net is 216000 lines of code, but this includes database
documentation, the source for this manual, and more.  The source code is
certainly above 120000 lines by any metric though.} While
it is not possible to cover the entire application here, a brief overview
of the source code is in order.

In the root of the install directory, one will find a setup.pl program,
a number of other .pl programs, and a number of directories. The setup.pl
program is used to update or install LedgerSMB. The other .pl programs
provide a basic set of services for the framework (including authentication)
and then pass the work on to the data entry screen file in the bin
directory.

The bin directory contains the workflow scripts inherited from SQL-Ledger.
These do basic calculations and generate the user interface.  The scripts/
directory does the same with new code.

The css directory in the root install directory contains CSS documents
to provide various stylesheets one can select for changing various
aspects of the look and feel of the application.

The locale directory contains translation files that LedgerSMB uses
to translate between different languages. One could add translations
to these files if necessary.

The UI directory contains user interface templates for old and new code.  These
templates use TemplateToolkit to generate the basic HTML.  The directory also
contains template-specific javascript and css files.

The LedgerSMB directory is where the Perl modules reside that provide the
core business logic in LedgerSMB. These modules provide functionality
such as form handling, email capabilities, and access to the database
through its at least partially object oriented API.

Finally, the sql directory provides the database schemas and upgrade
scripts.

\subsection{Data Entry Screens}

One can customize the data entry screens to optimize work flow, display
additional information, etc.


\subsubsection{Examples}

We set up hot keys for payment lines, automatically focused the keyboard
on the last partnumber field, removed separate print and post buttons
to ensure that invoices were always printed and posted together, and
removed the ability to print to the screen, and even the ability to
scan items in when an invoice was received (using a portable data
terminal) and import this data into LedgerSMB. Finally we added the
ability to reconcile the till online in a paperless manner.

For another customer, we added the ability to print AR invoices in
plain text format and added templates (based on the POS sales template)
to do this.


\subsection{Extensions}

One can add functionality to the Perl modules in the LSMB directory
and often add missing functions easily.


\subsubsection{Examples}

For one customer, we added a module to take data from a portable data
terminal collected when inventory items were taken and use that to
add shrinkage and loss adjustments. We also extended the parts model
to add a check id flag (for alcohol sales) and added this flag to
the user interface.

For another customer, we added a complex invoice/packing slip tracking
system that allowed one to track all the printed documents associated
with an order or invoice.


\subsection{Templates}

As noted before templates can be modified or extended, though sometimes
this involves extending the user interface scripts. Most templates
are easy enough to modify. 


\subsubsection{Examples}

For one customer we added text-only invoices for AR and AP transactions/Invoices
and an ability to use Javascript in them to automatically print them
on load.


\subsection{Reports}

The fact that all the data is available within the database manager
is a huge advantage of LedgerSMB over Quickbooks and the like. The
rapid development of reports allows for one to easily develop reports
of any sort within LedgerSMB.


\subsubsection{Examples}

For one customer, we developed a report of parts sold and received
during arbitrary time frames. The report allows one to go back and
look up the invoices involved.


\section{Integration Possibilities}

An open database system and programming API allows for many types
of integration. There are some challenges, but in the end, one can
integrate a large number of tools.


\subsection{Reporting Tools}

Any reporting tool which can access the PostgreSQL database can be
used with LedgerSMB for custom reporting. These can include programs
like Microsoft Access and Excel (using the ODBC drivers), PgAccess
(A PostgreSQL front-end written in TCL/Tk with a similar feel to Access),
Rekall, Crystal Reports, OpenOffice and more.


\subsubsection{Examples}

We have created spreadsheets of the summaries of activity by day and
used the ODBC driver to import these into Excel. Excel can also read
HTML tables, so one can use PostgreSQL to create an HTML table of
the result and save it with a .xls extension so that Windows opens
it with Excel. These could then be served via the same web server
that serves LedgerSMB.


\subsection{Line of Business Tools on PostgreSQL}

Various line of business tools have been written using PostgreSQL
in large part due to the fact that it is far more mature than MySQL
in areas relating to data integrity enforcement, transactional processing,
and the like. These tools can be integrated with LedgerSMB in various
ways. One could integrate this program with the HERMES CRM framework,
for example.

\subsubsection{Strategies}

In general, it is advisable to run all such programs that benefit
from integration in the same database but under different schemas.
This allows PostgreSQL to become the main method of synchronizing
the data in real time. However, sometimes this can require dumping
the database, recreating the tables etc. in a different schema and
importing the data back into LedgerSMB.

One possibility for this sort of integration is to use database triggers
to replicate the data between the applications in real-time. This
can avoid the main issue of duplicate id's. One issue that can occur
however relates to updates. If one updates a customer record in HERMES,
for example, how do we know which record to update in LedgerSMB?
There are solutions to this problem but they do require some forethought.

A second possibility is to use views to allow one application to present
the data from the other as its own. This can be cleaner regarding
update issues, but it can also pose issues regarding duplicate id
fields.


\subsubsection{Examples}

Others have integrated L'ane POS and LedgerSMB in order to make it
work better with touch screen devices. Still others have successfully
integrated LedgerSMB and Interchange. In both cases, I believe that
triggers were used to perform the actual integration.


\subsection{Line of Business Tools on other RDBMS's}

Often there are requests to integrate LedgerSMB with applications
like SugarCRM, OSCommerce, and other applications running on MySQL
or other database managers. This is a far more complex field and it
requires a great deal more effort than integrating applications within
the same database.


\subsubsection{Strategies}

Real-time integration is not always possible. MySQL does
not support the SQL extension SQL/MED (Management of External Data) that
supports non-MySQL data sources so it is not possible to replicate the 
data in real-time. Therefore
one generally resorts to integrating the system using time-based updates.
Replication may be somewhat error-prone unless the database manager
supports triggers (first added to MySQL in 5.0) or other mechanisms
to ensure that all changed records can be detected and replicated.
In general, it is usually advisable to add two fields to the record--
one that shows the insert time and one that shows the last update.

Additionally, I would suggest adding additional information to the
LedgerSMB tables so that you can track the source record from the
other application in the case of an update.

In general, one must write replication scripts that dump the information
from one and add it to the other. This must go both ways.


\subsubsection{Integration Products and Open Source Projects}

While many people write Perl scripts to accomplish the replication,
an open source project exists called DBI-Link. This package requires
PL/Perl to be installed in PostgreSQL, and it allows PostgreSQL to
present any data accessible via Perl's DBI framework as PostgreSQL
tables. DBI-Link can be used to allow PostgreSQL to pull the data
from MySQL or other database managers.

DBI-Link can simplify the replication process by reducing the operation
to a set of SQL queries.


\section{Customization Guide}

This section is meant to provide a programmer with an understanding
of the technologies enough information to get up to speed quickly
and minimize the time spent familiarizing themselves with the software.
Topics in this section are listed in order of complexity. As it appeals
to a narrower audience than previous discussions of this topic, it
is listed separately.


\subsection{General Information}

The main framework scripts (the ar.pl, ap.pl, etc. scripts found in
the root of the installation directory) handle such basic features
as instantiating the form object, ensuring that the user is logged
in, and the like. They then pass the execution off to the user interface
script (usually in the bin/mozilla directory).

LedgerSMB in many ways may look sort of object oriented in its design,
but in reality, it is far more data-driven than object oriented. The
Form object is used largely as a global symbol table and also as a
collection of fundamental routines for things like database access.
It also breaks down the query string into sets of variables which
are stored in its attribute hash table.

In essence one can and often will store all sorts of data structures
in the primary Form object. These can include almost anything. It
is not uncommon to see lists of hashes stored as attributes to a Form
object.

\subsection{Customizing Templates}

Templates are used to generate printed checks, invoices, receipts,
and more in LedgerSMB. Often the format of these items does not fit
a specific set of requirements and needs to be changed. This document
will not include \LaTeX{} or HTML instruction, but will include a
general introduction to editing templates. Also, this is not intended
to function as a complete reference.

Template instructions are contained in tags \textless?lsmb and ?\textgreater.
The actual parsing is done by the parse\_template function in LSMB/Form.pm.


\subsubsection{Template Control Structures}

As of 1.3, all templates use the Template Toolkit syntax for generating LaTeX,
text, and html output.  The LaTeX can then be processed to create postscript or
pdf files, and could be trivially extended to allow for DVI output as well.

Template Toolkit provides a rich set of structures for controlling flow which
are well beyond what was available in previous versions of LedgerSMB.  The only
difference is in the start and end tag sequences, where we use <?lsmb and ?> in
order to avoid problems with rendering LaTeX templates for testing purposes.

\subsection{Customizing Forms}

Data entry forms and other user interface pieces are in the bin directory.
In LedgerSMB 1.0.0 and later, symlinks are not generally used.

Each module is identified with a two letter combination: ar, ap, cp,
etc. These combinations are generally explained in the comment headers
on each file.

Execution in these files begins with the function designated by the
form->$\lbrace$action$\rbrace$ variable. This variable is usually
derived from configuration parameters in the menu.ini or the name
of the button that was clicked on to submit the previous page. Due
to localization requirements, the following process is used to determine
the appropriate action taken:

The \$locale-\textgreater getsub routine is called. This routine
checks the locale package to determine if the value needs to be translated
back into an appropriate LSMB function. If not, the variable is lower-cased,
and all spaces are converted into underscores.

In general there is no substitute for reading the code to understand
how this can be customized and how one might go about doing this.

In 1.3, all UI templates for new code (referenced in the scripts directory) and
even some for the old code (from the bin directory) use TemplateToolkit and 
follow the same rules as the other templates.

\subsection{Customizing Modules}

The Perl Modules (.pm files) in the LedgerSMB directory contain the
main business logic of the application including all database access.
Most of the modules are relatively easy to follow, but the code quality 
leaves something to be desired.  The API calls themselves are 
likely to be replaced in the future, so work on documenting API calls is
now focused solely on those calls that are considered to be stable.
At the moment, the best place to request information on the API's is on
the Developmers' Email List (\mailto{ledger-smb-devel@lists.sourceforge.net}).

Many of these modules have a fair bit of dormant code in them which
was written for forthcoming features, such as payroll and bills of
materials.

One can add a new module through the normal means and connect it to
other existing modules.


\subsubsection{Database Access}

Both the Form (old code) and LedgerSMB (new code) objects have a dbh property
which is a database handle to the PostgreSQL database.  This handle does not
autocommit.

\clearpage 


\part{Appendix}

\appendix
%dummy comment inserted by tex2lyx to ensure that this paragraph is not empty



\section{Where to Go for More Information}

There are a couple of relevant sources of information on LedgerSMB
in particular.

The most important resources are the LedgerSMB web site 
(\url{http://www.ledgersmb.org}) and the email lists found at our Sourceforge
site.

In addition, it is generally recommended that the main bookkeeper
of a company using LedgerSMB work through at least one accounting
textbook. Which textbook is not as important as the fact that a textbook
is used.


\section{Quick Tips}


\subsection{Understanding Shipping Addresses and Carriers}

Each customer can have a default shipping address. This address is
displayed prominantly in the add new customer screen. To change the
shipping address for a single order, one can use the ship to button
at the bottom of the quote, order, or invoice screen.

The carrier can be noted in the Ship Via field. However, this is a
freeform field and is largely used as commentary (or instructions
for the shipping crew).


\subsection{Handling bad debts}

In the event that a customer's check bounces or a collection requirement
is made, one can flag the customer's account by setting the credit
limit to a negative number.

If a debt needs to be written off, one can either use the allowance
method (by writing it against the contra asset account of \char`\"{}Allowance
for Bad Debts\char`\"{} or using the direct writeoff method where
it is posted as an expense.


\section{Step by Steps for Vertical Markets}


\subsection{Common Installation Errors}

\begin{itemize}
\item LedgerSMB is generally best installed in its own directory outside
of the wwwroot directory. While it is possible to install it inside
the wwwroot directory, the instructions and the faq don't cover the
common problems here. 
\item When the chart of accounts (COA) is altered such that it is no longer
set up with appropriate items, you can make it impossible to define
goods and services properly. In general, until you are familiar with
the software, it is best to rename and add accounts rather than deleting
them. 
\end{itemize}

\subsection{Retail With Light Manufacturing}

For purposes of this example we will use a business that assembles
computers and sells them on a retail store. 

\begin{enumerate}
\item Install LedgerSMB.
\item Set preferences, and customize chart of accounts.

\begin{enumerate}
\item Before customizing the COA it is suggested that you consult an accountant. 
\end{enumerate}
\item Define Goods, Labor, and Services as raw parts ordered from the vendors. 

\begin{itemize}
\item These are located under the goods and services menu node. 
\end{itemize}
\item Define assemblies.

\begin{itemize}
\item These are also located under goods and services. 
\item Component goods and services must be defined prior to creating assembly.
\end{itemize}
\item Enter an AP Invoice to populate inventory with proper raw materials. 

\begin{itemize}
\item One must generally add a generic vendor first. The vendor is added
under AP-\textgreater Vendors-\textgreater Add Vendor. 
\end{itemize}
\item To pay an AP invoice like a check, go to cash-$>$payment. Fill out 
approrpiate fields and click print. 

\begin{itemize}
\item Note that one should select an invoice and enter in the payment amount
in the appropriate line of the invoice list. If you add amounts to
the master amount list, you will find that they are added to the amount
paid on the invoice as a prepayment. 
\item The source field is the check number. 
\end{itemize}
\item Stock assemblies 
\item One can use AR Invoices or the POS interface to sell goods and services. 

\begin{itemize}
\item Sales Invoice 

\begin{itemize}
\item Can be generated from orders or quotations 
\item Cannot include labor/overhead except as part of an assembly 
\item One can make the mistake of printing the invoice and forgetting to
post it. In this event, the invoice does not officially exist in the
accounting system. 
\item For new customers, you must add the customer first (under AR-\textgreater
Customers-\textgreater Add Customer. 
\end{itemize}
\item POS Interface 

\begin{itemize}
\item Cannot include labor/overhead except as part of an assembly 
\item Printing without posting is often even easier in the POS because of
the rapid workflow. Yet it is just as severe a problem. 
\end{itemize}
\item Ecommerce and Mail Order Operations 

\begin{itemize}
\item See the shipping workflow documentation above. 
\end{itemize}
\item Customers are set up by going to AR-\textgreater Customers-\textgreater
Add Customer (or the equivalent localized translation). The appropriate
fields are filled out and the buttons are used at the bottom to save
the record and optionally use it to create an invoice, etc. 

\begin{itemize}
\item Saving a customer returns to the customer screen. After the appropriate
invoice, transaction, etc. is entered and posted, LedgerSMB will
return to the add customer screen. 
\end{itemize}
\end{itemize}
\item One can use the requirements report to help determine what parts need
to be ordered though one cannot generate PO's directly from this report.
\end{enumerate}
Note, the needs of LedgerSMB are mostly useful for light manufacturing
operations (assembling computers, for example). More manufacturing
capabilities are expected to be released in the next version.

A custom assembly is a bit difficult to make. One must add the assembly
prior to invoice (this is not true of goods and services). If the
assembly is based on a different assembly but may cost more (due to
non-standard parts) you can load the old assembly using Goods and
Services-\textgreater Reports-\textgreater Assemblies and then make
necessary changes (including to the SKU/Partnumber) and save it as
new.

Then one can add it to the invoice.


\section{Glossary}

\begin{description}
\item [{BIC}] Bank Identifier Code is often the same as the S.W.I.F.T.
code. This is a code for the bank a customer uses for automated money
transfers. 
\item [{COGS}] is Cost of Goods Sold. When an item is sold, then the expense
of its purchase is accrued as attached to the income of the sale.
It is tracked as COGS. 
\item [{Credit}] : A logical transactional unit in double entry accounting.
It is the opposite of a debit. Credits affect different account types
as follows: 

\begin{description}
\item [{Equity}] : Credits are added to the account when money is invested
in the business. 
\item [{Asset}] : Credits are added when money is deducted from an asset
account. 
\item [{Liability}] : Credits are added when money is owed to the business
account. 
\item [{Income}] : Credits are added when income is earned. 
\item [{Expense}] : Credits are used to apply adjustments at the end of
accounting periods to indicate that not all the expense for an AP
transaction has been fully accrued. 
\end{description}
\item [{Debit}] : A logical transactional unit in double entry accounting
systems. It is the opposite of a credit. Debits affect different account
types as follows: 

\begin{description}
\item [{Equity}] : Debits are added when money is paid to business owners. 
\item [{Asset}] : Debits are added when money is added to an account. 
\item [{Liability}] : Debits are added when money that is owed is paid
off. 
\item [{Income}] : Debits are used to temporarily adjust income to defer
unearned income to the next accounting period. 
\item [{Expense}] : Debits are added as expenses are incurred. 
\end{description}
\item [{IBAN}] International Bank Account Number is related to the BIC
and is used for cross-border automated money transfers. 
\item [{List}] Price is the recommended retail price. 
\item [{Markup}] is the percentage increase that is applied to the last
cost to get the sell price. 
\item [{ROP}] Re-order point. Items with fewer in stock than this will
show up on short reports. 
\item [{Sell}] Price is the price at which the item is sold. 
\item [{SKU}] Stock Keeping Unit: a number designating a specific product.
\item [{Source}] Document : a paper document that can be used as evidence
that a transaction occurred. Source documents can include canceled
checks, receipts, credit card statements and the like. 
\item [{Terms}] is the number of days one has to pay the invoice. Most
businesses abbreviate the terms as Net n where n is the number of
days. For example, Net 30 means the customer has 30 days to pay the
net due on an invoice before it is late and incurs late fees.  Sometimes
you will see 2 10 net 30 or similar.  This means 2\% discount if paid within
10 days but due within 30 days in any case.
\end{description}
\end{document}
